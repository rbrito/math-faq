\section{Formula for the Surface Area of a sphere in Euclidean $N$-Space}
This is equivalent to the volume of the $N-1$ solid which comprises the
boundary of an $N$-Sphere.

The volume of a ball is the easiest formula to remember: It's $r^N
\frac{\pi^{N/2}}{(N/2)!}$.  The only hard part is taking the factorial
of a half-integer.  The real definition is that $x! = \Gamma(x+1)$, but
if you want a formula, it's:

\[
(1/2+n)! = \sqrt{\pi} \frac{(2n+2)!}{(n+1)!4^{n+1}}
\]

To get the surface area, you just differentiate to get
$N\frac{\pi^{N/2}}{(N/2)!}r^{N-1}$.

There is a clever way to obtain this formula using Gaussian
integrals. First, we note that the integral over the line of $e^{-x^2}$
is $\sqrt{\pi}$.  Therefore the integral over $N$-space of
$e^{-x_1^2-x_2^2-...-x_N^2}$ is $\sqrt{\pi}^n$.  Now we change to
spherical coordinates.  We get the integral from 0 to infinity of
$Vr^{N-1}e^{-r^2}$, where $V$ is the surface volume of a sphere.
Integrate by parts repeatedly to get the desired formula.

It is possible to derive the volume of the sphere from ``first
principles''.
%%% Local Variables:
%%% mode: latex
%%% TeX-master: "math-faq"
%%% End:
