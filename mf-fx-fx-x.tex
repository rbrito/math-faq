\section{Name for $f(x)^{f(x)}=x$}


Solving for $f$ one finds a ``continued fraction''-like answer

\begin{equation}
    f(x) = \frac{\log x}{\log
               \frac{\log x}{\log
                \frac{\log x}{\log\ldots } } }
\end{equation}

This question has been repeated here from time to time over the years,
and no one seems to have heard of any published work on it, nor a
published name for it.

%From: asimov@nas.nasa.gov (Daniel A. Asimov)
%Date: Tue, 21 Nov 1995 13:50:24 -0800
%comments on the name....

This function is the inverse of $f(x)=x^x$. It might be argued that such
description is good enough as far as mathematical names go: ``the
inverse of the function $f(x)=x^x$'' seems to be clear and succint.

Another possible name is $lx(x)$. This comes from the fact that the
inverse of $e^x$ is $ln(x)$ thus the inverse of $x^x$ could be named
$lx(x)$.

%From: asimov@nas.nasa.gov (Daniel A. Asimov)
%Date: Tue, 21 Nov 1995 13:50:24 -0800
%Dan claims it is. Check
%What's more, it most certainly IS an analytic function
%(away from the singularities of x^x, of course).  For example,
%a branch of it is real analytic on the interval ((1/e)^(1/e), oo) in R,
%and complex analytic on some neighborhood of that interval in C.


It's not an analytic function.

The ``continued fraction'' form for its numeric solution is highly
unstable in the region of its minimum at $1/e$ (because the graph is
quite flat there yet logarithmic approximation oscillates wildly),
although it converges fairly quickly elsewhere. To compute its value
near $1/e$, use the bisection method which gives good results. Bisection
in other regions converges much more slowly than the logarithmic
continued fraction form, so a hybrid of the two seems suitable.  Note
that it's dual valued for the reals (and many valued complex for
negative reals).

A similar function is a built-in function in MAPLE called $W(x)$ or
Lambert's $W$ function.  MAPLE considers a solution in terms of $W(x)$
as a closed form (like the erf function). W is defined as
$W(x)e^{W(x)}=x$.

Notice that $f(x) = exp(W(log(x)))$ is the solution to $f(x)^f(x)=x$

An extensive treatise on the known facts of Lambert's W function is
available for anonymous ftp at
\url{ftp://dragon.uwaterloo.ca/cs-archive/CS-93-03/W.ps.Z}.
%%% Local Variables: 
%%% mode: latex
%%% TeX-master: "math-faq"
%%% End: 
