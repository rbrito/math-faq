\section{Indiana bill sets the value of $\pi$ to 3}


The bill {\it House Bill No. 246, Indiana State Legislature, 1897},
reportedly set the value of $\pi$ to an incorrect rational approximation.

The following is the text of the bill:

\begin{quote}
HOUSE BILL NO. 246


  "A bill for an act introducing a new mathematical truth and offered as a
contribution to education to be used only by the State of Indiana free of
cost by paying any royalties whatever on the same, provided it is accepted
and adopted by the official action of the legislature of 1897.

   "Section 1. Be it enacted by the General Assembly of the State of Indiana:
It has been found that a circular area is to the square on a line equal to the
quadrant of the circumference, as the area of an equilateral rectangle is to
the square on one side. The diameter employed as the linear unit according to
the present rule in computing the circle's area is entirely wrong, as it
represents the circles area one and one-fifths times the area of a square
whose perimeter is equal to the circumference of the circle. This is because
one-fifth of the diameter fils to be represented four times in the circle's
circumference. For example: if we multiply the perimeter of a square by
one-fourth of any line one-fifth greater than one side, we can, in like
manner make the square's area to appear one fifth greater than the fact, as
is done by taking the diameter for the linear unit instead of the quadrant
of the circle's circumference.

   "Section 2. It is impossible to compute the area of a circle on the
diameter as the linear unit without trespassing upon the area outside the
circle to the extent of including one-fifth more area than is contained within
the circle's circumference, because the square on the diameter produces the
side of a square which equals nine when the arc of ninety degrees equals
eight. By taking the quadrant of the circle's circumference for the linear
unit, we fulfill the requirements of both quadrature and rectification of
the circle's circumference. Furthermore, it has revealed the ratio of the
chord and arc of ninety degrees, which is as seven to eight, and also the
ratio of the diagonal and one side of a square which is as ten to seven,
disclosing the fourth important fact, that the ratio of the diameter and
circumference is as five-fourths to four; and because of these facts and the
further fact that the rule in present use fails to work both ways
mathematically, it should be discarded as wholly wanting and misleading in
its practical applications.


   "Section 3. In further proof of the value of the author's proposed
contribution to education, and offered as a gift to the State of Indiana,
is the fact of his solutions of the trisection of the angle, duplication of
the cube and quadrature having been already accepted as contributions to
science  by the American Mathematical Monthly, the leading exponent of
mathematical thought in this country. And be it remembered that these
noted problems had been long since given up by scientific bodies as
unsolvable mysteries and above man's ability to comprehend."
\end{quote}

Will E. Edington in an article published in the Proceedings of the
Indiana Academy of Science
 describes the fate of the bill in the committees of the Indiana
legislature. First it was referred to the House Committee on Canals, which was
also referred to as the Committee on Swamp Lands. Notices of the bill appeared
in the Indianapolis Journal and the Indianapolis Sentinel on Jan. 19, 1897,
both of which described it a a bill telling how to square circles. On the same
day, "Representative M.B.Butler, of Steuben County, chairman of the
Committee on Canals, submitted the following report:

\begin{quote}
 "Your Committee on Canals, to which was referred House Bill No.246, entitled
an act for the introduction of a mathematical truth, etc., has had the same
under consideration and begs leave to report the same back to the House with
the recommendation that said bill be referred to the Committee on Education."
\end{quote}

\noindent The next day, the following article appeared in the Indianapolis Sentinel:

\begin{quote}
                        "To SQUARE THE CIRCLE

   "Claims Made That This Old Problem Has Been Solved.
   "The bill telling how to square a circle, introduced in the House by
Mr.Record, is not intended to be a hoax. Mr. Record knows nothing of the bill
with the exception that he introduced it by request of Dr.Edwin Goodwin of
Posey County, who is the author of the demonstration. The latter and State
Superintendent of Public Instruction Geeting believe that it is the long-sought
solution of the problem, and they are seeking to have it adopted by the
legislature. Dr. Goodwin, the author, is a mathematician of note. He has it
copyrighted and his proposition is that if the legislature will indorse the
solution, he will allow the state to use the demonstration in its textbooks
free of charge. The author is lobbying for the bill."

On "February 2, 1897, ...Representative S.E. Nicholson, of Howard County,
chairman of the Committee on Education, reported to the House.


   "Your Committee on Education, to which was referred House Bill No.246,
entitled a a bill for an act entitled an act introducing a new mathematical
truth, has had same under consideration, and begs leave to report the same
back to the House with the recommendation that said bill do pass.

"The report was concurred in, and on February 8, 1897, it was brought up for 
the
second reading, following which it was considered engrossed. Then
'Mr. Nicholson moved that the constitutional rule requiring bills to be read
on three days be suspended, that the bill may be read a third time now.' The
constitutional rule was suspended by a vote of 72 to 0 and the bill was then
read a third time. It was passed by a vote of 67  to 0, and the Clerk of the
House was directed to inform the Senate of the passage of the bill."

\end{quote}

The newspapers reported the suspension of the consitutional rules and
the unanimous passage of the bill matter-of-factly, except for one line
in the Indianapolis Journal to the effect that "this is the strangest
bill that has ever passed an Indiana Assembly."



The bill was referred to the Senate on Feb.10, 1897, and was read for the first
time on Feb.11 and referred to the Committee on Temperance. "On Feb.12
Senator Harry S. New, of Marion County, Chairman of the Committee on
Temperance, made the following report to the Senate:


\begin{quote}
   "Your committee on Temperance, to which was referred House Bill No.246,
introduced by Mr.Record, has had the same under consideration and begs leave
to report the same back to the Senate with the recommendation that said bill
do pass."
\end{quote}

The Senate Journal mentions only that the bill was read a second time on
Feb.12, 1897, that there was an unsuccessful attempt to amend the bill
by strike out the enacting clause, and finally it was postponed indefinitely.
That the bill was killed appears to be a matter of dumb luck rather than the
superior education or wisdom of the Senate. It is true that the bill was
widely ridiculed in Indiana and other states, but what actually brought about
the defeat of the bill is recorded by Prof. C.A. Waldo in an article he wrote
for the Proceedings of the Indiana Academy of Science in 1916. The reason
he knows is that he happened to be at the State Capitol lobbying for the
appropriation of the Indiana Academy of Science, on the day the Housed passed
House Bill 246. When he walked in the found the debate on House Bill 246
already in progress. In his article, he writes (according to Edington):

\begin{quote}
"An ex-teacher from the eastern part of the state was saying: 'The case is
perfectly simple. If we pass this bill which establishes a new and correct
value for $\pi$, the author offers to our state without cost the use of his
discovery and its free publication in our school text books, while everyone
else must pay him a royalty.'"
\end{quote}

 The roll was then called and the bill passed its
third and final reading in the lower house. A member then showed the writer
[i.e. Waldo] a copy of the bill just passed and asked him if he would like
an introduction to the learned doctor, its author. He declined the courtesy
with thanks remarking that he was acquainted with as many crazy people as he
cared to know.

"That evening the senators were properly coached and shortly thereafter as it
came to its final reading in the upper house they threw out with much
merriment the epoch making discovery of the Wise Man from the Pocket."


The bill implies four different values
for $\pi$ and one for $\sqrt{2}$, as follows:
$\pi^\prime = 16/\sqrt{3}$, $2\sqrt{5\pi/6}$, $16\sqrt{2}/7$, $16/5
    (~9.24     , ~3.236      , ~3.232     , 3.2$ respectively.)
$\sqrt{2}^\prime = 10/7.$

\begin{quote}
 It has been found that a circular area is to the square on a line
 equal to the quadrant of the circumference, as the area of an
 equilateral rectangle is to the square on one side.
\end{quote}
$\pi^\prime  : {(\pi^\prime/2)}^2 = \sqrt{3}/4 : 1$
i.e. $\pi^\prime = 16/\sqrt{3}           ~= 9.24$.


\begin{quote}
 The diameter employed as the linear unit according to the present rule
 in computing the circle's area is entirely wrong, as it represents the
 circles area one and one-fifths times the area of a square whose
 perimeter is equal to the circumference of the circle. This is because
 one-fifth of the diameter fails to be represented four times in the
 circle's circumference.
\end{quote}

Bit tricky to decipher, but it seems to say ${(2\pi^\prime /4)}^26/5 = \pi$
i.e. $\pi^\prime = 2 \sqrt{5\pi/6}        ~= 3.236$

\begin{quote}
 Furthermore, it has revealed the ratio of the chord and arc of ninety
degrees, which is as seven to eight,
\end{quote}

$\sqrt{2} : \pi/2 = 7 : 8$
i.e. $\pi = 16 \sqrt{2}/7          ~= 3.232$

\begin{quote}
and also the ratio of the diagonal and one side of a square which is
as ten to seven
\end{quote}

i.e. $\sqrt{2} = 10/7             ~= 1.429$


\begin{quote}
that the ratio of the diameter and circumference is as five-fourths to
four
\end{quote}

i.e. $\pi = 16/5                  = 3.2$



