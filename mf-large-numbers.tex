\section{Names of Large Numbers}


Naming for $10^k$.
\begin{verbatim}

k     American      European    SI--Prefix


-33                                revo
-30                                tredo
-27                                syto
-24                                fito
-21                                ento
-18   quintillionth                atto
-15   Quadrillionth                femto
-12   trillionth                   pico
-9    Billionth                    nano
-6    Millionth                    micro
-3    Thousandth                   milli
-2    Hundredth                    centi
-1    Tenth                        deci
1     Ten                          deca
2     Hundred                      hecto
3     Thousand                     kilo
4     Myriad 
6     Million       Million        mega
9     Billion       Milliard       giga   
12    Trillion      Billion        tera
15    Quadrillion   Billiard       peta
18    Quintillion   Trillion       exa
21    Sextillion    Trilliard      hepa
24    Septillion    Quadrillion    otta
27    Octillion     Quadrilliard   nea	
30    Nonillion     Quintillion    dea	
      (Noventillion)
33    Decillion     Quintilliard   una
36    Undecillion          Sextillion
39    Duodecillion         Sextilliard
42    tredecillion         Septillion
45     quattuordecillion   Septilliard
48   quindecillion         Octillion
51   sexdecillion          Octilliard
54   septendecillion       Nonillion
                           (Noventillion)
57   octodecillion         Nonilliard
                           (Noventilliard)
60   novemdecillion        Decillion
63  VIGINTILLION           Decilliard

6*n   (2n-1)-illion n-illion
6*n+3 (2n)-illion   n-illiard

100   Googol        Googol
303   CENTILLION
600                 CENTILLION

10^100 Googolplex   Googolplex

%From: balden@wimsey.com (Bruce Balden) 
%Date: Fri, 11 Oct 1996 12:46:39 GMT 

Chinese System


1       yi4
10      shi2
100     bai3
1000     qian2
10000   wan4
10^6     yi bai3 wan (i.e. 100 times wan)
10^8     yi1
10^12   ???
               


The American system is used in:
      US,
      ...

The European system is used in:
      Austria,
      Belgium,
      Chile,
      Germany,
      the Netherlands,
      Italy (see exception)
      Scandinavia


%Date: Mon, 25 Aug 1997 22:47:48 -0700
%From: Torbjorn Larsson <ekatla@eka.ericsson.se>
%Subject: Sci.math FAQ                  

Note that all prefixes are to be spelled with a leading small letter. (As are
all SI units, even those that honors persons by using their names.)

- All prefixes with n < 0 should have a small letter abbreviation.
Eg. 1 picoampere = 1 pA. (SI unit rule explanation: person name unit
is abreviated using a capital letter)

- All prefixes with n > 0 should have a large letter abbreviation.
Eg. 1 gigameter = 1 Gm. (SI unit rule explanation: non-person name unit
is abreviated in lower case).  _Except_ the mass unit: 1 kilogram is
abreviated as kg (compare to Km. for kilometer).


hv@cix.compulink.co.uk (Hugo van der Sanden):
  To the best of my knowledge, the House of Commons decided to adopt the
  US definition of billion quite a while ago - around 1970? - since which
  it has been official government policy.

dik@cwi.nl (Dik T. Winter):
  The interesting thing about all this is that originally the French used
  billion to indicate 10^9, while much of the remainder of Europe used
  billion to indicate 10^12.  I think the Americans have their usage from
  the French.  And the French switched to common European usage in 1948.

gonzo@ing.puc.cl (Gonzalo Diethelm):
  Other countries (such as Chile, my own, and I think
  most of Latin America) use billion to mean 10^12, trillion to mean
  10^18, etc. What is the usage distribution over the world population,
  anyway?

\end{verbatim}
