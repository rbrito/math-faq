\section{A Few Words on ``False'' Proofs}

It is quite frequent for one to be presented with some false proof where
one absurd or obviously wrong fact is derived. The intention of this
section is to alert that it is easy to write a wrong proof for a false
statement, as is frequently the case when certain discussions come up.

For instance, if one were allowed to divide by zero, then we could prove
that $1 = 2$, in just a few steps:
\begin{align}
  x^2 - x^2 &= x^2 - x^2 \rightarrow \\
  x(x-x) &= (x+x)(x-x) \rightarrow \\
  x &= x+x \rightarrow \\
  x &= 2x \rightarrow \\
  1 &= 2
\end{align}

Similar ``proofs'' abound using unsound arguments, like using the false
fact that if the square of two real numbers is the same, then the
squared numbers are the same.

\subsection{False Also Implies True}

In fact, it is also possible to present a wrong ``proof'' for a true
fact, as is the case presented on many public forums where proofs are
shown. This, by the way, highlights the need to be skeptical when
reading such proofs, especially when ``miraculous'' conclusions are
drawn from ``clever arguments''.

Let's suppose that $1 = 2$. Then, of course, $2 = 1$. Summing each
equation sidewise, we obtain $3 = 3$. This shows that a false hypothesis
can lead to a true conclusion. In other words, when devising a proof to
a fact, care must be taken to ensure that every single step is sound or,
otherwise, the \emph{whole} argument is flawed.

This is a particular fact of the truth table of implication, shown
below, for some given statements:
\begin{tabular}{c|c|c}
  $p$ & $q$ & $p \Rightarrow q$\\
  \hline
  true & true & true\\
  true & false & false\\
  false & true & false\\
  false & false & false
\end{tabular}

A manner to read the table above is, for the first line...