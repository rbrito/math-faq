\section{Fields Medal}

\subsection{Historical Introduction}

This is the original letter by Fields creating the endowment for the
medals that bear his name.
It is thought to have been written during the few months before his death.
Notice that no mention is made about the age of the recipients
(currently there is a 40 year-old limit), and that the medal should
not be attached to any person, private or public, meaning that it
shouldn't bear anybody's name.

Colloquially, people refer to the Fields Medal as ``the Nobel Prize for
Mathematics''. This is not quite correct, as the age restriction 
makes it impossible to award the prize to distinguished mathematicians
who are very deserving, such as Andrew Wiles. 

\begin{quote}
  It is proposed to found two gold medals to be awarded at successive
International Mathematical Congress for outstanding achievements in
mathematics. Because of the multiplicity of the branches of mathematics
and taking into account the fact that the interval between such congresses
is four years it is felt that at least two medals should be available. The
awards would be open to the whole world and would be made by an
International Committee.

  The fund for the founding of the medals  is constituted by balance left
over after financing the Toronto congress held in 1924. This must be held
in trust by the Government or by some body authorized by government to hold
and invest such funds. It would seem that a dignified method for handling
the matter and one which in this changing world should most nearly secure
permanency would be for the Canadian Government to take over the fund and
appoint as his custodian say the Prime Minister of the Dominion or the
Prime Minister in association with the Minister of Finance. The medals
would be struck at the Mint in Ottawa and the duty of the custodian would
be simply to hand over the medals at the proper time to the accredited
International Committee.

  As things are at present a practical course of procedure would seem to be
for the Executive Committee of a Congress to appoint a small international
committee authorized to add to its number and call into consultation other
mathematicians as it might deem expedient. The Committee would be expected
to decide on the ones to whom the awards should be made thirty months in
advance of the following Congress. Its decisions would be communicated to
the President and Secretary of the Organizing Committee of the Congress,
this Committee having the duty of communicating to the Prime Minister of
Canada the names of the recipients in order that the medal might be
prepared in time and forwarded to the president of the Organizing
Committee. Immediately on the appointment of the Executive Committee of the
Congress the medals would be handed over to its President. The presentation
of the medals would constitute a special feature at some general meeting of
the Congress.

  In the above arrangements the role of the Organizing Committee might be
taken over by the Executive of the International Mathematical Union at some
time in the future when that organization has been generally accepted.

  In coming to its decision the hands of the IC should be left as free as
possible. It would be understood, however, that in making the awards while
it was in recognition of work already done it was at the same time intended
to be an encouragement for further achievement on the part of the
recipients and a stimulus to renewed effort on the part of others.

 In commenting on the work of the medalists it might be well to be
conservative in one's statements to avoid envidious comparisons explicit or
implied. The Committee might ease matters by saying they have decided to
make the awards along certain lines not alone because of the outstanding
character of the achievement but also with a view to encouraging further
development along these lines. In this connection the Committee might say
that they had elected to select subjects in Analysis, in Geometry, in the
Theory of Groups, in the Theory of Numbers etc. as the case might be. When
the Committee had come to an agreement in this sense the claims for
recognition of work done along the special lines in question could be
considered in detail by two smaller groups or subcommittees with
specialized qualifications who would have authority to take into
consultation or add to the subcommittees other mathematicians of
specialized knowledge.

  With regard to the medals themselves, I might say that they should each
contain at least 200 dollars worth of gold and be of a fair size, probably
7.5 centimeters in diameter. Because of the international character the
language to be employed it would seem should be Latin or Greek? The design
has still to be definitely determined. It will have to be decided on by
artists in consultation with mathematicians. The suggestions made in the
preceding are tentative and open to consideration on the part of
mathematicians.

  It is not contemplated to make an award until 1936 at the Congress
following that at Zurich during which an international Medal Committee
should be named.

  The above programme means a new departure in the matter of international
scientific cooperation and is likely to be the precursor of moves along
like lines in other sciences than mathematics.

  One would hear again emphasized the fact that the medals should be of a
character as purely international and impersonal as possible. There should
not be attached to them in any way the name of any country, institution or
person.

  Perhaps provision could be made as soon as possible after the appointment
of the Executive of the Zurich Congress for the consideration by it of the
subject of the medals, and the appointment without undue delay of a
Committee and the awards of the medals to be made in connection with the
Congress of 1936.

  Suggestions with regard to the design of the medals will be welcome.


                                 (signed)   J.C. Fields
                                            Research Professor of Mathematics
                                            University of Toronto
\end{quote}


More information may also be found at
%
%From Joel Chan
%Date: Sat, 25 Nov 1995 22:54:20 -0500
%on your next update could you change the URL
%
\url{http://www.math.toronto.edu/fields.html}

\vfill\pagebreak

\subsection{Table of Awardees}

\begin{tabular}{|r|l|l|l|r|} \hline

{Year} & {Name} & {Birthplace} & {Country}& {Age} 
 \\ \hline

1936&Ahlfors, Lars     &Helsinki      &Finland  &29\\
1936&Douglas, Jesse    &New York, NY   &USA      &39\\
1950&Schwartz, Laurent &Paris         &France   &35\\
1950&Selberg, Atle     &Langesund     &Norway   &33\\
1954&Kodaira, Kunihiko &Tokyo         &Japan    &39\\ 
1954&Serre, Jean-Pierre&Bages         &France   &27\\
1958&Roth, Klaus       &Breslau       &Germany  &32\\
1958&Thom, Rene        &Montbeliard   &France   &35\\
1962&Hormander, Lars   &Mjallby       &Sweden   &31\\
1962&Milnor, John      &Orange, NJ     &USA      &31\\
1966&Atiyah, Michael   &London        &UK       &37\\
1966&Cohen, Paul       &Long Branch NJ&USA      &32\\
1966&Grothendieck, Alex.&Berlin   &Germany  &38\\
1966&Smale, Stephen    &Flint, MI      &USA      &36\\
1970&Baker, Alan       &London        &UK       &31\\
1970&Hironaka, Heisuke &Yamaguchi-ken &Japan    &39\\
1970&Novikov, Serge    &Gorki         &USSR     &32\\
1970&Thompson, John    &Ottawa, KA     &USA      &37\\
1974&Bombieri, Enrico  &Milan         &Italy    &33\\
1974&Mumford, David    &Worth, Sussex &UK       &37\\
1978&Deligne, Pierre   &Brussels      &Belgium  &33\\
1978&Fefferman, Charles&Washington DC &USA      &29\\
1978&Margulis, Gregori &Moscow        &USSR     &32\\
1978&Quillen, Daniel   &Orange, NJ     &USA      &38\\
1982&Connes, Alain     &Draguignan    &France   &35\\
1982&Thurston, William &Washington DC &USA      &35\\
1982&Yau, Shing-Tung   &Kwuntung      &China    &33\\
1986&Donaldson, Simon  &Cambridge     &UK       &27\\
1986&Faltings, Gerd    &1954          &Germany  &32\\
1986&Freedman, Michael &Los Angeles &USA      &35\\
\hline
\end{tabular}

\begin{tabular}{|r|l|l|l|r|} \hline

{Year} & {Name} & {Birthplace} & {Country}& {Age} 
 \\ \hline

1990&Drinfeld, Vladimir&Kharkov       &USSR     &36\\
1990&Jones, Vaughan    &Gisborne      &N Zealand&38\\
1990&Mori, Shigefumi   &Nagoya        &Japan    &39\\
1990&Witten, Edward    &Baltimore     &USA      &38\\
1994& Pierre-Louis Lions    & ????  &   France  &   38\\
1994&Jean-Chrisophe Yoccoz  &   ????  &   France  &   37\\
%From: Elly Gustafsson <gustafss@IAS.EDU>
%Date: Tue, 19 Mar 1996 16:37:31 -0500
%Birthplace:   Oostende
%Country:      Belgium
%Institution:  Institute for Advanced Study   only
1994&Jean Bourgain          &  Oostende  &   Belge   &   40\\
%Date: Tue, 23 Apr 1996 12:13:09 -0400
%From: Dmitriy Rumynin <rumynin@math.umass.edu>
%The place of birth of Efim Zelmanov was Novosibirsk.
1994&Efim Zelmanov          &   Novosibirsk  &   Russia  &   39\\
\hline
\end{tabular}

\bigskip
\bigskip

\begin{tabular}{|r|l|l|l|} \hline

{Year} & {Name} & {Institution} &{Country}
 \\ \hline

1936&Ahlfors, Lars     &Harvard University         &USA \\
1936&Douglas, Jesse    &MIT              &USA \\
1950&Schwartz, Laurent &Universite de Nancy       &France \\
1950&Selberg, Atle     &Institute for Advanced Study, Princeton&USA \\
1954&Kodaira, Kunihiko &Princeton University      &USA \\
1954&Serre, Jean-Pierre&College de France&France \\
1958&Roth, Klaus       &University of London      &UK \\
1958&Thom, Rene        &University of Strasbourg  &France \\
1962&Hormander, Lars   &University of Stockholm   &Sweden \\
1962&Milnor, John      &Princeton University      &USA \\
1954&Serre, Jean-Pierre&College de France&France \\
1958&Roth, Klaus       &University of London      &UK \\
1958&Thom, Rene        &University of Strasbourg  &France \\
1962&Hormander, Lars   &University of Stockholm   &Sweden \\
1962&Milnor, John      &Princeton University      &USA \\
1966&Atiyah, Michael   &Oxford University         &UK \\
1966&Cohen, Paul       &Stanford University       &USA \\
1966&Grothendieck, Alex&University of Paris       &France \\
1966&Smale, Stephen    &University of California at Berkeley      &USA \\
1970&Baker, Alan       &Cambridge University      &UK \\
1970&Hironaka, Heisuke &Harvard University        &USA \\
1970&Novikov, Serge    &Moscow University         &USSR \\
1970&Thompson, John    &University of Chicago     &USA \\
1974&Bombieri, Enrico  &Univeristy of Pisa        &Italy \\
1974&Mumford, David    &Harvard University        &USA \\
1978&Deligne, Pierre   &IHES             &France \\
1978&Fefferman, Charles&Princeton University      &USA \\
1978&Margulis, Gregori &InstPrblmInfTrans&USSR \\
1978&Quillen, Daniel   &MIT              &USA \\
1982&Connes, Alain     &IHES             &France \\
1982&Thurston, William &Princeton University      &USA \\
1982&Yau, Shing-Tung   &Institute for Advanced Study, Princeton              &USA \\
1986&Donaldson, Simon  &Oxford University         &UK \\
1986&Faltings, Gerd    &Princeton University      &USA \\
1986&Freedman, Michael &University of California at San Diego     &USA \\
\hline
\end{tabular}

\begin{tabular}{|r|l|l|l|} \hline

{Year} & {Name} & {Institution} &{Country}
 \\ \hline

1990&Drinfeld, Vladimir&Phys.Inst.Kharkov&USSR \\
1990&Jones, Vaughan    &University of California at Berkeley      &USA \\
1990&Mori, Shigefumi   &University of Kyoto?      &Japan \\
1990&Witten, Edward    &Princeton/Institute for Advanced Study&USA \\
1994& Lions, Pierre-Louis &Universite de Paris-Dauphine&France\\
  1994& Yoccoz, Jean-Chrisophe       &Universite de Paris-Sud&France\\
  1994& Bourgain, Jean               &Institute for Advanced Study&USA\\
  1994&  Zelmanov, Efim               &University of Wisconsin&USA\\
1998&  Borcherd, Richard         &Cambridge University&Great Britain \\
  1998& Gowers, William Timothy     &Cambridge University&Great Britain\\
  1998& Kontsevich, Maxim           &IHES Bures-sur-Yvette&Russia\\
  1998& McCullen, Curtis&Harvard University&USA\\
\hline
\end{tabular}


\bigskip
\bigskip


\Ref

\book{International Mathematical Congresses, An Illustrated History 1893-1986.}
     {Donald J.Alberts, G. L. Alexanderson and Constance Reid.}
     {Revised Edition, Including 1986,  Springer Verlag, 1987.}

\article{Tropp, Henry S.}
        {The origins and history of the Fields Medal.}
        {Historia Mathematica,} {3(1976), 167-181.}
