\section{Cutting a sphere into pieces of larger volume}


Is it possible to cut a sphere into a finite number of pieces and
reassemble into a solid of twice the volume?

This question has many variants and it is best answered explicitly.

Given two polygons of the same area, is it always possible to dissect
one into a finite number of pieces which can be reassembled into a
replica of the other?

Dissection theory is extensive.  In such questions one needs to specify

\begin{itemize}
  \item What is a ``piece''?  (polygon?  Topological disk?  Borel-set?
  Lebesgue-measurable set?  Arbitrary?)

  \item How many pieces are permitted (finitely many? countably?
  uncountably?)

  \item What motions are allowed in ``reassembling'' (translations?
  rotations?  orientation-reversing maps?  isometries?  affine maps?
  homotheties?  arbitrary continuous images?  etc.)

  \item How the pieces are permitted to be glued together.  The simplest
  notion is that they must be disjoint.  If the pieces are polygons [or
  any piece with a nice boundary] you can permit them to be glued along
  their boundaries, ie the interiors of the pieces disjoint, and their
  union is the desired figure.
\end{itemize}

Some dissection results

\begin{itemize}
  \item We are permitted to cut into finitely many polygons, to
  translate and rotate the pieces, and to glue along boundaries; then
  yes, any two equal-area polygons are equi-decomposable.

  This theorem was proven by Bolyai and Gerwien independently, and has
  undoubtedly been independently rediscovered many times.  I would not
  be surprised if the Greeks knew this.

  The Hadwiger-Glur theorem implies that any two equal-area polygons are
  equi-decomposable using only translations and rotations by $180$
  degrees.

  \item
  \begin{teo}[Hadwiger-Glur, 1951]
    Two equal-area polygons $P$,$Q$ are equi-decomposable by
    translations only, iff we have equality of these two functions:
    $\phi_P() = \phi_Q()$
  \end{teo}
  Here, for each direction $v$ (ie, each vector on the unit circle in
  the plane), let $\phi_P(v)$ be the sum of the lengths of the edges of
  $P$ which are perpendicular to $v$, where for such an edge, its length
  is positive if $v$ is an outward normal to the edge and is negative if
  $v$ is an inward normal to the edge.


  \item In dimension 3, the famous ``Hilbert's third problem'' is:
  \begin{quote}
    If $P$ and $Q$ are two polyhedra of equal volume, are they
    equi-decomposable by means of translations and rotations, by cutting
    into finitely many sub-polyhedra, and gluing along boundaries?
  \end{quote}

  The answer is {\bf no} and was proven by Dehn in 1900, just a few
  months after the problem was posed. (Ueber raumgleiche polyeder,
  Goettinger Nachrichten 1900, 345-354). It was the first of Hilbert's
  problems to be solved. The proof is nontrivial but does not use the
  axiom of choice.

  \Ref

  \book{Hilbert's Third Problem.}{V.G. Boltianskii.}{Wiley 1978.}


  \item Using the axiom of choice on non-countable sets, you can prove
  that a solid sphere can be dissected into a finite number of pieces
  that can be reassembled to two solid spheres, each of same volume of
  the original. No more than nine pieces are needed.

  The minimum possible number of pieces is five.  (It's quite easy to
  show that four will not suffice).  There is a particular dissection in
  which one of the five pieces is the single center point of the
  original sphere, and the other four pieces $A$, $A^\prime$, $B$,
  $B^\prime$ are such that $A$ is congruent to $A^\prime$ and $B$ is
  congruent to $B^\prime$.  [See Wagon's book].

  This construction is known as the {\em Banach-Tarski paradox} or the
  {\em Banach-Tarski-Hausdorff} paradox (Hausdorff did an early version
  of it).  The ``pieces'' here are non-measurable sets, and they are
  assembled disjointly (they are not glued together along a boundary,
  unlike the situation in Bolyai's thm.)  An excellent book on
  Banach-Tarski is:


  \book{The Banach-Tarski Paradox.}{Stan Wagon.}{Cambridge University
    Press, 985}

  \article{Robert M. French.}{The Banach-Tarski theorem.}{The
    Mathematical Intelligencer,}{10 (1988) 21-28.}


  The pieces are not (Lebesgue) measurable, since measure is preserved
  by rigid motion. Since the pieces are non-measurable, they do not have
  reasonable boundaries. For example, it is likely that each piece's
  topological-boundary is the entire ball.

  The full Banach-Tarski paradox is stronger than just doubling the
  ball.  It states:

  \item Any two bounded subsets (of $3$-space) with non-empty interior,
  are equi-decomposable by translations and rotations.

  This is usually illustrated by observing that a pea can be cut up into
  finitely pieces and reassembled into the Earth.

  The easiest decomposition ``paradox'' was observed first by Hausdorff:

  \item The unit interval can be cut up into countably many pieces
  which, by translation only, can be reassembled into the interval of
  length 2.

  This result is, nowadays, trivial, and is the standard example of a
  non-measurable set, taught in a beginning graduate class on measure
  theory.
  \item
  \begin{teo}
    There is a finite collection of disjoint open sets in the unit cube
    in $R^3$ which can be moved by isometries to a finite collection of
    disjoint open sets whose union is dense in the cube of size 2 in
    $R^3$.
  \end{teo}
  This result is by Foreman and Dougherty.

%Jorge Stolfi
%stolfi@dcc.unicamp.br

  \item A square {\bf cannot} be rearranged into a disk, if one is
  allowed finitely many pieces with analytic boundaries, glued at edges.
  \item A square can be rearranged into a disk, with translations only,
  if one is allowed to use finitely many pieces with unconstrained shape
  (not necessarily connected), and disjoint assembly.
\end{itemize}

\Ref

\article{Boltyanskii.}{Equivalent and equidecomposable figures.}  {in
  Topics in Mathematics published by D.C. HEATH AND CO., Boston.}{}

\article{Dubins, Hirsch and ?}{Scissor Congruence}{American Mathematical
  Monthly.}{}

``Banach and Tarski had hoped that the physical absurdity of this
theorem would encourage mathematicians to discard AC. They were dismayed
when the response of the math community was `Isn't AC great?  How else
could we get such counterintuitive results?'{}''
