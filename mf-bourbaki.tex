\section{Who is N. Bourbaki?}

A group of mostly French mathematicians which began meeting in the
1930s, aiming to write a thorough unified account of all
mathematics. They had tremendous influence on the way math is done
since. For a very accessible sampler see Dieudonne Mathematics: The
Music Of Reason (Orig. Pour L'honneur De L'esprit Humain).

The founding is described in Andre Weil's autobiography, titled
something like ``memoir of an apprenticeship'' (orig. Souvenirs
D'apprentissage). There is a usable book Bourbaki by J. Fang. Liliane
Beaulieu has a book forthcoming, which you can sample in ``A Parisian
Cafe and Ten Proto-Bourbaki Meetings 1934-1935'' in the Mathematical
Intelligencer 15 no.1 (1993) 27-35.

The history behind Bourbaki is also described in Scientific American,
May 1957.
