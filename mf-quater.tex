Four-dimensional analog to the theory of complex analytic functions.  It
was developed in the 1930s by the mathematician Fueter.  It is based on
a generalization of the Cauchy-Riemann equations, since the possible
alternatives of power series expansions or quaternion differentiability
do not produce useful theories. A number of useful integral theorems
follow from the theory. Sudbery provides an excellent review. Deavours
covers some of the same material less thoroughly.  Brackx discusses a
further generalization to arbitrary Clifford algebras.

\Ref

\article{Anthony Sudbery.}{Quaternionic
      Analysis.}{Proc. Camb. Phil. Soc.,} {vol. 85, pp 199-225, 1979.}

\article{Cipher A. Deavours.}{The Quaternion
      Calculus.}{Am. Math. Monthly,} {vol. 80, pp 995-1008, 1973.}

\book{Clifford analysis.}{F. Brackx and R. Delanghe and F. Sommen.}
       {Pitman, 1983.}
%%% Local Variables: 
%%% mode: latex
%%% TeX-master: "math-faq"
%%% End: 
