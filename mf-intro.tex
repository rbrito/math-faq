\section{Why a list of Frequently Asked Questions?}

\emph{The Net}, as users call the Internet, and specially
\emph{newsgroups}, (i.e. Usenet) created a demand of knowledge without
parallel since the invention of the printing press.  Surprisingly, the
type of knowledge demanded from and by the Usenet community had, in
most cases, little in common---both in structure and content---with
that of printed in current publications. This defined Usenet as more of an
alternative to books rather than a replacement thereof\footnote{It could be
argued that books fulfill their mandate and purpose to everybody's
satisfaction.  Thus, even though the Net could, in principle, replace
the need for books, people choose not to do so. Instead it's domain is
defined, by its very nature to be disjoint from books.}

In the Net, questions posed are, more often than not, at the level 
of an amateur practitioner --even in cases where the question
was posed by a professional in the field. Similarly, the quality of the
answers  varies greatly, ranging from the incorrect or disrespectful,
to summaries of the state of the art in the topic in question.

Other characteristics of communication on the Net are simply inherited
from restrictions of the medium. The unit of knowledge is a screenful
worth of text (a scrit, from screen and bit). Articles exceeding that
limit are usually disregarded.

The lack of memory of the medium generates a repetition of topics,
much to the chagrin of old time citizens of the Net. 
Frequently asked questions lists palliate some of these deficiencies by
providing a record of relevant information while at the same time 
never being outdated. 

Thus, typically a list of frequently asked questions is ``posted''
at least once a month, and updated at least as frequently. And,
in what must be a first for an information based product, FAQ lists 
``expire'' on a given date, very much like any other  perishable item.


\section{Frequently Asked Questions in Mathematics?}

If I had to describe the contents of the FAQ in Mathematics
in a single sentence, I would call it \emph{mathematical gossip}
or perhaps \emph{non-trivial mathematical trivia}.

The FAQ list is a compilation of knowledge of interest to most
professional and amateur mathematicians, ranging from advanced
topics such as Wiles' proposed proof to Fermat's Last Theorem
to the list of Fields Medal winners.
