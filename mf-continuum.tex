\section{The Continuum Hypothesis}
A basic reference is Godel's ``What is Cantor's Continuum Problem?'',
from 1947 with a 1963 supplement, reprinted in Benacerraf and Putnam's
collection Philosophy of Mathematics.  This outlines Godel's generally
anti-CH views, giving some ``implausible'' consequences of CH.

\begin{quote}
  ``I believe that adding up all that has been said one has good reason
  to suspect that the role of the continuum problem in set theory will
  be to lead to the discovery of new axioms which will make it possible
  to disprove Cantor's conjecture.''
\end{quote}

At one stage he believed he had a proof that $C = \aleph_2$ from some
new axioms, but this turned out to be fallacious.  (See Ellentuck,
``Godel's Square Axioms for the Continuum'', Mathematische Annalen
1975.)

Maddy's ``Believing the Axioms'', Journal of Symbolic Logic 1988 (in 2
parts) is an extremely interesting paper and a lot of fun to read.  A
bonus is that it gives a non-set-theorist who knows the basics a good
feeling for a lot of issues in contemporary set theory.

Most of the first part is devoted to ``plausible arguments'' for or
against CH: how it stands relative to both other possible axioms and to
various set-theoretic ``rules of thumb''.  One gets the feeling that the
weight of the arguments is against CH, although Maddy says that many
``younger members'' of the set-theoretic community are becoming more
sympathetic to CH than their elders.  There's far too much here for me
to be able to go into it in much detail.

Some highlights from Maddy's discussion, also incorporating a few things
that other people sent me:

\begin{enumerate}
  \item Cantor's reasons for believing CH aren't all that persuasive today.

  \item Godel's proof of the consistency of CH shows that CH follows
  from ZFC plus the Axiom of Constructibility ($V=L$, roughly that the
  set-theoretic universe = the constructible universe).  However, most
  set-theorists seem to find Constructiblity implausible and much too
  restrictive.  It's an example of a ``minimizing'' principle, which
  tends to cut down on the number of sets admitted to one's universe.
  Apparently ``maximizing'' principles meet with much more sympathy from
  set theorists.  Such principles are more compatible with $\neg$CH than
  with CH.

  \item If GCH is true, this implies that $\aleph_0$ has certain unique
  properties: e.g. that it's that cardinal before which GCH is false and
  after which it is true.  Some would like to believe that the
  set-theoretic universe is more ``uniform'' (homogeneous) than that,
  without this kind of singular occurrence.  Such a ``uniformity''
  principle tends to imply $\neg$GCH.

  \item Most of those who disbelieve CH think that the continuum is
  likely to have very large cardinality, rather than $\aleph_2$ (as
  Godel seems to have suggested).  Even Cohen, a professed formalist,
  argues that the power set operation is a strong operation that should
  yield sets much larger than those reached quickly by stepping forward
  through the ordinals:

  \begin{quote}
    ``This point of view regards C as an incredibly rich set given to us
    by a bold new axiom, which can never be approached by any piecemeal
    process of construction.''
  \end{quote}

  \item There are also a few arguments in favour of CH, e.g. there's an
  argument that $\neg$CH is restrictive (in the sense of (2) above).
  Also, CH is much easier to force (Cohen's method) than $\neg$CH.  And
  CH is much more likely to settle various outstanding results than is
  $\neg$CH, which tends to be neutral on these results.

  \item Most large cardinal axioms (asserting the existence of cardinals
  with various properties of hugeness: these are usually derived either
  from considering the hugeness of $\aleph_0$ compared to the finite
  cardinals and applying uniformity, or from considering the hugeness of
  V (the set-theoretic universe) relative to all sets and applying
  ``reflection'') don't seem to settle CH one way or the other.

  \item Various other axioms have some bearing.  Axioms of determinacy
  restrict the class of sets of reals that might be counterexamples to
  CH.  Various forcing axioms (e.g. Martin's axiom), which are
  ``maximality'' principles (in the sense of (2) above), imply $\neg$CH.
  The strongest (Martin's maximum) implies that $C = \aleph_2$.  Of
  course the ``truth'' or otherwise of all these axioms is
  controversial.

  \item Freiling's principle about ``throwing darts at the real line''
  is a seemingly very plausible principle, not involving large cardinals
  at all, from which $\neg$CH immediately follows.  Freiling's paper
  (JSL 1986) is a good read.  More on this at the end of this message.
\end{enumerate}

Of course we have conspicuously avoided saying anything about whether
it's even reasonable to suppose that CH has a determinate truth-value.
Formalists will argue that we may choose to make it come out whichever
way we want, depending on the system we work in.  On the other hand, the
mere fact of its independence from ZFC shouldn't immediately lead us to
this conclusion -- this would be assigning ZFC a privileged status which
it hasn't necessarily earned.  Indeed, Maddy points out that various
axioms within ZFC (notably the Axiom of Choice, and also Replacement)
were adopted for extrinsic reasons (e.g. ``usefulness'') as well as for
``intrinsic'' reasons (e.g. ``intuitiveness'').  Further axioms, from
which CH might be settled, might well be adopted for such reasons.

One set-theorist correspondent said that set-theorists themselves are
very loathe to talk about ``truth'' or ``falsity'' of such claims.
(They're prepared to concede that $2+2=4$ is true, but as soon as you
move beyond the integers trouble starts.  e.g. most were wary even of
suggesting that the Riemann Hypothesis necessarily has a determinate
truth-value.)  On the other hand, Maddy's contemporaries discussed in
her paper seemed quite happy to speculate about the ``truth'' or
``falsity'' of CH.

The integers are not only a bedrock, but also any finite number of power
sets seem to be quite natural Intuitively are also natural which would
point towards the fact that CH may be determinate one way or the other.
As one correspondent suggested, the question of the determinateness of
CH is perhaps the single best way to separate the Platonists from the
formalists.

And is it true or false?  Well, CH is somewhat intuitively plausible.
But after reading all this, it does seem that the weight of evidence
tend to point the other way.

The following is from Bill Allen on Freiling's Axiom of Symmetry.  This
is a good one to run your intuitions by.

\begin{quote}
  Let $A$ be the set of functions mapping Real Numbers into countable
  sets of Real Numbers.  Given a function $f$ in $A$, and some arbitrary
  real numbers $x$ and $y$, we see that $x$ is in $f(y)$ with
  probability 0, i.e. $x$ is not in $f(y)$ with probability 1.
  Similarly, $y$ is not in $f(x)$ with probability 1.  Let AX be the
  axiom which states

  ``for every $f$ in $A$, there exist $x$ and $y$ such that $x$ is not
  in $f(y)$ and $y$ is not in $f(x)$''

  The intuitive justification for AX is that we can find the $x$ and $y$
  by choosing them at random.

  In ZFC, AX = not CH.

  proof:

  If CH holds, then well-order $R$ as $r_0, r_1, .... , r_x, ...$ with
  $x < \aleph_1$.  Define $f(r_x)$ as $\{r_y : y \leq x\}$.  Then $f$ is
  a function which witnesses the falsity of AX.

  If CH fails, then let $f$ be some member of $A$.  Let $Y$ be a subset
  of $R$ of cardinality $\aleph_1$.  Then $Y$ is a proper subset.  Let
  $X$ be the union of all the sets $f(y)$ with $y$ in $Y$, together with
  $Y$.  Then, as $X$ is an $\aleph_1$ union of countable sets, together
  with a single $\aleph_1$ size set $Y$, the cardinality of $X$ is also
  $\aleph_1$, so $X$ is not all of $R$.  Let a be in $R \ X$, so that a
  is not in $f(y)$ for any $y$ in $Y$.  Since $f(a)$ is countable, there
  has to be some $b$ in $Y$ such that $b$ is not in $f(a)$.  Thus we
  have shown that there must exist $a$ and $b$ such that $a$ is not in
  $f(b)$ and $b$ is not in $f(a)$.  So AX holds.
\end{quote}

Freiling's proof, does not invoke large cardinals or intense infinitary
combinatorics to make the point that CH implies counter-intuitive
propositions.  Freiling has also pointed out that the natural extension
of AX is AXL (notation mine), where AXL is AX with the notion of
countable replaced by Lebesgue Measure zero.  Freiling has established
some interesting Fubini-type theorems using AXL.

See ``Axioms of Symmetry: Throwing Darts at the Real Line'', by
Freiling, Journal of Symbolic Logic, 51, pages 190-200.  An extension of
this work appears in ``Some properties of large filters'', by Freiling
and Payne, in the JSL, LIII, pages 1027-1035.

The section above was excerpted from a posting from David Chalmers, of
Indiana University.

See also

Nancy McGough's *Continuum Hypothesis article* or its *mirror*.

\url{http://www.jazzie.com/ii/math/ch/}

\url{http://www.best.com/~ii/math/ch/}
%%% Local Variables:
%%% mode: latex
%%% TeX-master: "math-faq"
%%% End:
