\section{What is $0^0$}

According to some Calculus textbooks, $0^0$ is an ``indeterminate
form''. When evaluating a limit of the form $0^0$, then you need to know
that limits of that form are called ``indeterminate forms'', and that
you need to use a special technique such as L'Hopital's rule to evaluate
them. Otherwise, $0^0=1$ seems to be the most useful choice for
$0^0$. This convention allows us to extend definitions in different
areas of mathematics that otherwise would require treating 0 as a
special case. Notice that $0^0$ is a discontinuity of the function
$x^y$. More importantly, keep in mind that the value of a function and
its limit need not be the same thing, and functions need not be
continous, if that serves a purpose (see Dirac's delta).

This means that depending on the context where $0^0$ occurs, you might
wish to substitute it with 1, indeterminate or undefined/nonexistent.

Some people feel that giving a value to a function with an essential
discontinuity at a point, such as $x^y$ at $(0,0)$, is an inelegant
patch and should not be done. Others point out correctly that in
mathematics, usefulness and consistency are very important, and that
under these parameters $0^0=1$ is the natural choice.

The following is a list of reasons why $0^0$ should be 1.

Rotando \& Korn show that if $f$ and $g$ are real functions that vanish
at the origin and are analytic at 0 (infinitely differentiable is not
sufficient), then $f(x)^{g(x)}$ approaches 1 as $x$ approaches 0 from
the right.

From Concrete Mathematics p.162 (R. Graham, D. Knuth, O. Patashnik):
\begin{quote}
  Some textbooks leave the quantity $0^0$ undefined, because the
  functions $x^0$ and $0^x$ have different limiting values when $x$
  decreases to 0. But this is a mistake. We must define $x^0 = 1$ for
  all $x$, if the binomial theorem is to be valid when $x=0$, $y=0$,
  and/or $x=-y$.  The theorem is too important to be arbitrarily
  restricted! By contrast, the function $0^x$ is quite unimportant.
\end{quote}
Published by Addison-Wesley, 2nd printing Dec, 1988.

As a rule of thumb, one can say that $0^0 = 1$, but $0.0^{0.0}$ is
undefined, meaning that when approaching from a different direction
there is no clearly predetermined value to assign to $0.0^{0.0}$; but
Kahan has argued that $0.0^{0.0}$ should be 1, because if $f(x), g(x)
\rightarrow 0$ as $x$ approaches some limit, and $f(x)$ and $g(x)$ are
analytic functions, then $f(x)^g(x) \rightarrow 1$.


% correct name of Schlomiclh's by Edgar Fuss April, 1997
The discussion on $0^0$ is very old, Euler argues for $0^0=1$ since
$a^0=1$ for $a \neq 0$. The controversy raged throughout the nineteenth
century, but was mainly conducted in the pages of the lesser journals:
Grunert's Archiv and Schlomilch's Zeitschrift f\"ur Mathematik und
Physik. Consensus has recently been built around setting the value of
$0^0=1$.

On a discussion of the use of the function $0^{0^x}$ by an Italian
mathematician named Guglielmo Libri.

\begin{quote}
  [T]he paper [33] did produce several ripples in mathematical waters
  when it originally appeared, because it stirred up a controversy about
  whether $0^0$ is defined.  Most mathematicians agreed that $0^0 = 1$,
  but Cauchy [5, page 70] had listed $0^0$ together with other
  expressions like $0/0$ and $\infty-\infty$ in a table of undefined
  forms.  Libri's justification for the equation $0^0 = 1$ was far from
  convincing, and a commentator who signed his name simply ``S'' rose to
  the attack [45].  August M\"obius [36] defended Libri, by presenting
  his former professor's reason for believing that $0^0 = 1$ (basically
  a proof that $\lim_{x\rightarrow 0+} x^x = 1$).  M\"obius also went
  further and presented a supposed proof that $\lim_{x\rightarrow 0+}
  f(x)^{g(x)}$ whenever $\lim_{x\rightarrow 0+} f(x) = lim_{x\rightarrow
    0+} g(x) = 0$.  Of course ``S'' then asked [3] whether M\"obius knew
  about functions such as $f(x) = e^{-1/x}$ and $g(x) = x$.  (And paper
  [36] was quietly omitted from the historical record when the collected
  words of M\"obius were ultimately published.)  The debate stopped
  there, apparently with the conclusion that $0^0$ should be undefined.

  But no, no, ten thousand times no!  Anybody who wants the binomial
  theorem $ (x+y)^n = \sum_{k=0}^n {n\choose k} x^k y^{n-k}$ to hold for
  at least one nonnegative integer $n$ {\it must} believe that $0^0 =
  1$, for we can plug in $x = 0$ and $y = 1$ to get 1 on the left and
  $0^0$ on the right.

  The number of mappings from the empty set to the empty set is $0^0$.
  It {\it has} to be 1.

  On the other hand, Cauchy had good reason to consider $0^0$ as an
  undefined {\it limiting form}, in the sense that the limiting value of
  $f(x)^{g(x)}$ is not known {\it a priori} when $f(x)$ and $g(x)$
  approach 0 independently.  In this much stronger sense, the value of
  $0^0$ is less defined than, say, the value of $0+0$.  Both Cauchy and
  Libri were right, but Libri and his defenders did not understand why
  truth was on their side.

  [3] \article{Anonymous and S$\ldots$}{Bemerkungen zu den Aufsatze
    \"uberschrieben, `Beweis der Gleichung $0^0 = 1$, nach
    J. F. Pfaff',}{im zweiten Hefte dieses Bandes, S. 134, Journal f\"ur
    die reine und angewandte Mathematik,}{12 (1834), 292--294.}

  [5] \book{\OE uvres Compl\`etes.}{Augustin-Louis Cauchy.}{Cours
    d'Analyse de l'Ecole Royale Polytechnique (1821).  Series 2, volume
    3.}

  [33] \article{Guillaume Libri.}{M\'emoire sur les fonctions
    discontinues, Journal f\"ur die reine und angewandte
    Mathematik,}{}{10 (1833), 303--316.}

  [36] \article{A. F. M\"obius.}{Beweis der Gleichung $0^0 = 1$, nach
    J. F.  Pfaff.}  {Journal f\"ur die reine und angewandte
    Mathematik,}{}{12 (1834), 134--136.}

  [45] \article{S$\ldots$}{Sur la valeur de $0^0$.}{Journal f\"ur die
    reine und angewandte Mathematik 11,}{(1834), 272--273.}
\end{quote}

\Ref

\article{Knuth.}{Two notes on notation.}{(AMM 99 no. 5 (May 1992),}
{403--422).}

\article{H. E. Vaughan.}{The expression '$0^0$'.}{Mathematics Teacher 63
  (1970),} {pp.111-112.}

% John W. Loux
% and  Bill Dubuque <wgd@martigny.ai.mit.edu>  

\article{Kahan, W.} {Branch Cuts for Complex Elementary Functions or
  Much Ado about Nothing's Sign Bit,}{The State of the Art in Numerical
  Analysis, editors A. Iserles and M. J. D. Powell, Clarendon Press,
  Oxford, pp. 165--212. } \article{Louis M. Rotando and Henry Korn.}{The
  Indeterminate Form $0^0$.}  {Mathematics Magazine,}{Vol. 50, No. 1
  (January 1977), pp. 41-42.}

\article{L. J. Paige,.}{A note on indeterminate forms.}{American
  Mathematical Monthly,}{61 (1954), 189-190; reprinted in the
  Mathematical Association of America's 1969 volume, Selected Papers on
  Calculus, pp. 210-211.}

\article{Baxley \& Hayashi.}{A note on indeterminate forms.}{American
  Mathematical Monthly,}{85 (1978), pp. 484-486.}

\article{\ }{Crimes and Misdemeanors in the Computer Algebra Trade.}
{Notices of the American Mathematical Society,}{ September 1991, volume
  38, number 7, pp.778-785}
%%% Local Variables: 
%%% mode: latex
%%% TeX-master: "math-faq"
%%% End: 
