\section{Algebraic structures}

We will attempt to give a brief explanation of the following concepts:
\begin{itemize}
  \item \N is a monoid
  \item \Z is an integral domain
  \item \Q is a field
  \item in the field \R the order is complete
  \item the field \C is algebraically complete
\end{itemize}

If you have been asked by a child to give them arithmetic problems, so
they could show off their newly learned skills in addition and
subtraction I'm sure that after a few problems such as: $2 + 3$, $9 -
5$, $10 + 2$ and $6 - 4$, you tried tossing them something a little more
difficult: $4 - 7$ only to be told ``{\it That's not allowed.}''

What you may not have realized is that you and the child did not just
have different objects in mind (negative numbers) but entirely different
{\it algebraic systems}. In other words a set of objects (they could be
natural numbers, integers or reals) and a set of operations, or rules
regarding how the numbers can be combined.

We will take a very informal tour of some algebraic systems, but before
we define some of the terms, let us build a structure which will have
some necessary properties for examples and counterexamples that will
help us clarify some of the definitions.

We know that any number that is divided by six will either leave a
remainder, or will be divided exactly (which is after all the remainder
0). Let us write any number by the remainder $n$ it leaves after
division by six, denoting it as $[ n ]$. This means that, $7, 55$ and
$1$ will all be written $[1]$, which we call the {\it class} to which
they all belong: i.e. $7 \in [1]$, $55 \in [1]$, or, a bit more
technically, they are all equivalent to 1 modulo 6.  The complete set of
class will contain six elements, and this is called partitioning numbers
into equivalent classes because it separates (or partitions) all of our
numbers into these classes, and any one number in a class is equivalent
to any other in the same class.

One interesting thing we can do with these classes is to try to add or
to multiply them. What can $[1] + [3]$ mean? We can, rather naively try
out what they mean in ``normal'' arithmetic: $[1] + [3] = [1+3] = [4]$.
So far so good, let us try a second example $25 \in [1]$ and $45 \in
[3]$, their sum is $70$ which certainly belongs to $[4]$. Here we see
what we meant above by equivalence, 25 is equivalent to 1 as far as this
addition is concerned. Of course this is just one example, but
fortunately it can be proven that the sum of two classes is always the
class of the sums.

Now this is the kind of thing we all do when we add hours for example, 7
(o' clock) plus 6 hours is 1 (o' clock), and all we are really doing is
adding hours (modulo 12).

The neat part comes with multiplication, as we will see later on. But
for now just remember, it can be proven that something like $[4] \times
[5] = [2]$ will work: the product of two classes is the class of the
product.

Now for some of the necessary terminology.

\subsection{Monoids and Groups}

We need to define a {\em group}.

Let us take a set of objects and a rule (called a binary operation)
which allows us to combine any two elements of this set. Addition is an
example from math, or ANDing in some computer language.

The set must be closed under the operation. That means that when two
elements are combined the result must also be in the set. For example
the set containing even numbers will always give us an even number when
two elements are added together. But if we restrict ourselves to odd
numbers, their sum is not an odd number and so we know right off the bat
that the set of odd numbers and addition cannot constitute a group. Some
books will consider closure in the definition of binary operation, and
others add it as one of the requirements for a group along with the ones
that follow below.

The set and the operation is called a group if the binary operation
satisfies the following criteria:
\begin{itemize}
  \item the operation is associative, which means it doesn't matter how
  you group the elements you are operating on, for example in our set of
  remainders: $[1] + ([3] + [4]) = ([1] + [3]) + [4]$
  \item there is an identity element, meaning: one of the elements
  combined with the others in the set doesn't change them in the
  least. For example the zero in addition, or the one in multiplication.
  \item every element has an inverse with respect to that operation. If
  you combine an element and its inverse you get the identity (of that
  operation) back.
\end{itemize}

(Be careful with this last one, $-3$ is the inverse of $3$ in addition,
since they give us 0 when added, but $1/3$ is the inverse of 3 with
respect to multiplication, since $3 \times 1/3 = 1$ the identity under
multiplication.)

So we can see that the set of natural numbers \N (with the operation of
addition) is not even a group, since there is no inverse for 5, for
example.  (In other words there is no natural number which added to 5
will give us zero.) And so the third rule for our operation is violated.
But it still has \emph{some} structure, even if it is not as rich as the
ones we'll see later on.

Sets with an associative operation (the first condition above) are
called semigroups, and if they also have an identity element (the second
condition) then they are called monoids.

Our set of natural numbers under addition is then an example of a
monoid, a structure that is not quite a group because it is missing the
requirement that every element have an inverse under the operation
(Which is why in elementary school $4 - 7$ is not allowed.)

What about the set of integers, is it a group?

By itself this question is nonsensical. Why? Well, we have not mentioned
under what operation. OK, let us say: the set of integers with addition.

Now, addition is associative, the zero does not change any number when
added to it, and for every number $n$ we can add $-n$ and get zero. So
it's a group all right.

In fact it is a special kind of group. When we can perform the operation
on the two elements in any order (e.g $a + b = b + a$) then the group is
called commutative, or {\em Abelian} in honor of Abel. Not every
operation is commutative, for example three minus two is certainly {\em
  not} the same as two minus three. Our set of integers under addition
is then an Abelian group.


\subsection{Rings}

If we take an Abelian group (remember: a set with a binary operation)
and we define a second operation on it we get a bit more of a structure
than we had with just a group.

If the second operation is associative, and it is distributive over the
first then we have a ring. Note that the second operation may not have
an identity element, nor do we need to find an inverse for every element
with respect to this second operation. As for what distributive means,
intuitively it is what we do in math when perform the following change:
$a \times (b + c) = (a \times b) + (a \times c)$.

If the second operation is also commutative then we have what is called
a commutative ring. The set of integers (with addition and
multiplication) is a commutative ring (with even an identity---called
unit element---for multiplication).

Now let us go back to our set of remainders. What happens if we multiply
$[5] \times [1]$? We see that we get $[5]$, in fact we can see a number
of things according to our definitions above, $[5]$ is its own inverse,
and $[1]$ is the multiplicative element. We can also show easily enough
(by creating a complete multiplication table) that it is
commutative. But notice that if we take $[3]$ and $[2]$, neither of
which are equal to the class that the zero belongs to $[0]$, and we
multiply them, we get $[3] \times [2] = [0]$.  This bring us to the next
definition. In a commutative ring, let us take an element which is not
equal to zero and call it $a$. If we can find a non-zero element, say
$b$ that combined with $a$ equals zero ($a \times b = 0$) then $a$ is
called a {\it zero divisor}.

A commutative ring is called an integral domain if it has no zero
divisors.  Well the set \Z{} with addition and multiplication fullfills
all the necessary requirements, and so it is an integral domain. Notice
that our set of remainders is not an integral domain, but we can build a
similar set with remainders of division by five, for example, and
voil\`a, we have an integral domain.

Let us take, for example, the set \Q{} of rational numbers with addition
and multiplication - I'll leave out the proof that it is a ring, but I
think you should be able to verify it easily enough with the above
definitions. But to give you a head start, notice the addition of
rationals follow all the requirements for an abelian group. If we remove
the zero we will have another abelian group, and that implies that we
have something more than a ring, in fact, as we will see in the next
section.

\subsection{Fields}

Now we can make one step further. If the elements of a ring, excluding
the zero, form an abelian group (with the second operation) then it is a
field.  For example, write the multiplication table of the remainders of
division by 5, and you will see that it satisfies all the requirements
for a group: (You will probably have noticed that the group does not
contain the number five itself since $[5] = [0]$.)
\[
\begin{tabular}{  c  |  c  c  c  c  }
    & 1 & 2 & 3 & 4  \\
    \hline
  1 & 1 & 2 & 3 & 4 \\
  2 & 2 & 4 & 1 & 3 \\
  3 & 3 & 1 & 4 & 2 \\
  4 & 4 & 3 & 2 & 1
\end{tabular}
\]

(Why isn't the set of divisors of six---excluding the zero and under
multiplication---a group?  That's easy enough, since we have excluded
the zero we do not have the result of $[2] \times [3]$ in our set, so it
isn't closed.)

\subsection{Ordering}

Given a ring, we can say that it is ordered when you have a special
subset of that ring behaves in a very special way. If any two elements
of that special subset are added or multiplied their sum and their
product are again in the special subset. Take the negative numbers in
\R{}, can they be that special subset? Well the sum seems to be
allright, it is also a negative number. But things don't work with the
product: it is positive. What about the positive numbers? Yep, and in
fact we call that special subset, the set of positive elements. Now, we
gave the definition for an ordered ring, we can also define an ordered
field the same way.

But what does a complete ordered field mean? Well the definition looks
rather nasty: it is complete if every non-empty subset which posesses an
upper bound has a least upper bound.

Let's translate some of that, trying to lose as little information on
the way. A bound is something that guarantees that all of the elements
of your set are on one side of it (reasonably enough). For example,
certainly all negative reals are less than 100, so 100 is a bound (it is
in fact an upper bound 'cause all negatives are ``below'' it). But there
are lots of other bounds, 1, 5, 26 will all do nicely. The question now
is, of all of these (upper bounds) which is the smallest, that is which
one is ``the border'' so to speak? Does it always exists?

Let's take the rationals, and look at the following numbers:
\[1.4, \; 1.41, \; 1.414, \; 1.4142, \; 1.41421, \ldots \]

Now each of these is a rational number (it can be written as a
fraction), and they are getting closer and closer to a number we've
probably seen before (just take out your calculator and find the square
root of two).  So we can write the shorthand for this series as
$\sqrt{2}$.  Certainly we can find an upper bound for this series, $3$
will do nicely, but so can $1.5$, or $1.42$. But what is the
smallest. Well there isn't any. Not among the rational at least, because
no matter what fraction you give I can give you one closer to the square
root of two. What about the square root of two itself? Well it's not a
rational number (I'll skip the proof, but it is really rather easy) so
you can't use it.  If you want another series which is really neat look
at the section on ``Euler's formula'' in the FAQ.

And that is where the reals come in. Any set or reals that is bounded
you can certainly find the smallest of these bounds. (By the way this
``least upper bound'' is abbreviated ``l.u.b.'', or ``sup'' for {\it
  supremum}.) We can also turn things around and talk of lower bounds,
and of the largest of these etc. but most of that will be just a mirror
image of what we have dealt with so far.

So that should be it. And for years that did seem to be it, we seemed to
have all the numbers we'd ever care to have.

There was just one small stick in the works, but most people just sort
of pretended not to notice, and that was that not all polynomials had
solutions. One simple polynomial of this kind is $x^2 + 1 = 0$.  It's so
simple, yet there's no self respecting number that would solve this
polynomial. There were these funny answers which seemed like they should
be solutions but no one could make any sense out of them, so they were
considered imaginary solutions.  Which was really too bad because they
were given the name of imaginary numbers and now that the name stuck we
realize that they are numbers just as good as any of the ones we have
been using for centuries. And in fact that takes us to the last great
pinnacle in this short excursion. The field of complex numbers.

We can define an algebraically closed field as a field where every
nonconstant polynomial (i.e. one with an $x$ in it from high school
days) has a zero in the field. Whew! This in short means that as long as
the polynomial is not a constant number (which is no fun anyways) but
something which looks like it wants a solution, like $5 x^3 - 2 x^2 + 6
= 0$ it will always have one, if you are working with complex numbers
and not just reals.


There is another definition which is probably just as good, but may or
may not be easier: A field is algebraically closed if every polynomial
splits into linear factors.  Linear factors are briefly factors not
containing $x$ to any power of two or higher, in other words in the
form: $ax + b$. For example $x^2 + x -6$ can be factored as $(x + 3)(x -
2)$, but if we are in the field of reals we cannot factor $x^2 + 1$, but
we can in the field of complex numbers: $x^2 + 1 = (x - i)(x + i)$,
where, you may recall, $i^2 = -1$.
%%% Local Variables:
%%% mode: latex
%%% TeX-master: "math-faq"
%%% End:
