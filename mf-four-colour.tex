\section{The Four Colour Theorem}

\bigskip

\begin{teo}[Four Colour Theorem]
  Every planar map with regions of simple borders can be coloured with
  $4$ colours in such a way that no two regions sharing a non-zero
  length border have the same colour.
\end{teo}

An equivalent combinatorial interpretation is
\begin{teo}[Four Colour Theorem]
  Every loopless planar graph admits a vertex-colouring with at most
  four different colours.
\end{teo}

This theorem was proved with the aid of a computer in 1976.  The proof
shows that if aprox. 1,936 basic forms of maps can be coloured with four
colours, then any given map can be coloured with four colours. A
computer program coloured these basic forms. So far nobody has been able
to prove it without using a computer. In principle it is possible to
emulate the computer proof by hand computations.

The known proofs work by way of contradiction. The basic thrust of the
proof is to assume that there are counterexamples, thus there must be
minimal counterexamples in the sense that any subset of the graphic is
four colourable. Then it is shown that all such minimal counterexamples
must contain a subgraph from a set basic configurations.

But it turns out that any one of those basic counterexamples can be
replaced by something smaller, while preserving the need for five
colours, thus contradicting minimality.

The number of basic forms, or configurations has been reduced to 633.

A recent simplification of the Four Colour Theorem proof, by Robertson,
Sanders, Seymour and Thomas, has removed the cloud of doubt hanging over
the complex original proof of Appel and Haken.

The programs can now be obtained by ftp and easily checked over for
correctness. Also the paper part of the proof is easier to verify. This
new proof, if correct, should dispel all reasonable criticisms of the
validity of the proof of this theorem.

\Ref

\article{K. Appel and W. Haken.}  {Every planar map is four colorable.}
{Bulletin of the American Mathematical Society,} {vol. 82, 1976
  pp.711-712.}

\article{K. Appel and W. Haken.}  {Every planar map is four colorable.}
{Illinois Journal of Mathematics,} {vol. 21, 1977, pp. 429-567.}

%From: weemba@sagi.wistar.upenn.edu (Matthew P Wiener)
%Date: Sun, 10 Dec 1995 10:24:32 -0500 (EST)
% ftp info...

\article{N. Robertson, D. Sanders, P. Seymour, R. Thomas} {The Four
  Colour Theorem} {Preprint,}{ February 1994. Available by anonymous ftp
  from \url{ftp://ftp.math.gatech.edu/pub/users/thomas/fcdir/npfc.ps}.}

\book{The Four Color Theorem: Assault and Conquest} {T. Saaty and Paul
  Kainen.}  {McGraw-Hill, 1977. Reprinted by Dover Publications 1986.}
