
\section{Fermat's Last Theorem}


\subsection{History of Fermat's Last Theorem}

Pierre de Fermat (1601-1665) was a lawyer and amateur
mathematician. In about 1637, he annotated his copy (now lost)
of Bachet's translation of Diophantus' Arithmetika with the
following statement:

\begin{quote} {\it
Cubum autem in duos cubos, aut quadratoquadratum in duos
quadratoquadratos, et  generaliter nullam in infinitum ultra quadratum
potestatem in duos ejusdem  nominis fas est dividere: cujus rei
demonstrationem mirabilem sane detexi. Hanc marginis exiguitas non
caperet.}
\end{quote}
In English, and using modern terminology, the paragraph above reads
as: 
\begin{quote}
There are no positive integers such that $x^n + y^n = z^n$ for $n>2$.
I've found a remarkable proof of this fact, but there is not 
enough space in the margin [of the book] to write it.
\end{quote}
Fermat never published a proof of this statement. It became to be
known as Fermat's Last Theorem (FLT) not because it was his last piece of
work, but because it is the last remaining statement in
the post-humous list of Fermat's works that needed to be
proven or independently verified. All others have either been
shown to be true or disproven long ago. 




\subsection{What is the current status of FLT?}

\bigskip

    \begin{teo}
[Fermat's Last Theorem]
    There are no positive integers $x$, $y$, $z$, and $n > 2$ such that 
$x^n + y^n = z^n$.
\end{teo}
    Andrew Wiles, a researcher at Princeton, claims to have
    found a proof.  The proof was presented in Cambridge, UK during a
    three day seminar to an audience which included some of the
    leading experts in the field.  The proof was found to be wanting.
    In summer 1994, Prof. Wiles acknowledged that a gap
    existed. On October 25th, 1994, Prof. Andrew Wiles released two
    preprints, {\em Modular elliptic curves and Fermat's Last
    Theorem}, by Andrew Wiles, and {\em Ring theoretic properties of
    certain Hecke algebras}, by Richard Taylor and Andrew Wiles.
The first one (long) announces a proof of, among other things, Fermat's
Last Theorem, relying on the second one (short) for one crucial step.

The argument described by Wiles in his Cambridge
lectures had a serious gap, namely the construction of an
Euler system.  After trying unsuccessfully to repair that construction,
Wiles went back to a different approach he had tried earlier but
abandoned in favor of the Euler system idea.  He was able to complete his
proof, under the hypothesis that certain Hecke algebras are local complete
intersections.  This and the rest of the ideas described in Wiles'
Cambridge lectures are written up in the first manuscript.  Jointly,
Taylor and Wiles establish the necessary property of the Hecke
algebras in the second paper.

The new approach turns out to be significantly simpler and
shorter than the original one, because of the removal of the Euler system.
(In fact, after seeing these manuscripts Faltings has apparently come up
with a further significant simplification of that part of the argument.)

% As indicated by Dana W. Albrecht dwa@corsair.com. Jul 21/1995.
% Correction added Aug 21/1995

The papers were published in the May 1995 issue of {\it Annals of Mathematics}.
For single copies of the issues send e-mail to jlorder@jhunix.hcf.jhu.edu
for further directions.

In summary:

Both manuscripts have been published. Thousands of people have a read them. 
About a hundred understand it very well. Faltings has simplified
the argument already. Diamond has generalized it. People can read
it. The immensely complicated geometry has mostly been replaced by
simpler algebra. The proof is now generally accepted. There was a gap
in this second proof as well, but it has been filled since
October 1994. 

%Apparently retired without replacement, as noted by:
%From: wdr@world.std.com (William D Ricker)
%Date: Tue, 28 Nov 1995 13:37:56 -0500
%
%You may also peruse the AMS site on Fermat's Last Theorem at:
%
%\begin{verbatim}
%gopher://e-math.ams.org/11/lists/fermat
%\end{verbatim}

\CC{\subsection{Wiles' line of attack}

Here is an outline of the {\bf first, incorrect} proposed proof.
The bits about Euler system are 

\begin{itemize}

    \item {\bf From Ken Ribet:}

    Here is a brief summary of what Wiles said in his three lectures.

    The method of Wiles borrows results and techniques from lots and lots
    of people.  To mention a few: Mazur, Hida, Flach, Kolyvagin, yours
    truly, Wiles himself (older papers by Wiles), Rubin...  The way he does
    it is roughly as follows.  Start with a mod $p$ representation of the
    Galois group of $Q$ which is known to be modular.  You want to prove that
    all its lifts with a certain property are modular.  This means that the
    canonical map from Mazur's universal deformation ring to its maximal
    Hecke algebra quotient is an isomorphism.  To prove a map like this is
    an isomorphism, you can give some sufficient conditions based on
    commutative algebra.  Most notably, you have to bound the order of a
    cohomology group which looks like a Selmer group for $Sym^2$ of the
    representation attached to a modular form.  The techniques for doing
    this come from Flach; and then the proof went on to use Euler systems a la
    Kolyvagin, except in some new geometric guise. [This part turned out
    to be wrong and unnecessary].
    
    If you take an elliptic curve over $Q$, you can look at the
    representation of Gal on the 3-division points of the curve.  If you're
    lucky, this will be known to be modular, because of results of Jerry
    Tunnell (on base change).  Thus, if you're lucky, the problem I
    described above can be solved (there are most definitely some
    hypotheses to check), and then the curve is modular.  Basically, being
    lucky means that the image of the representation of Galois on
    3-division points is $GL(2,Z/3Z)$.
    
    Suppose that you are unlucky, i.e., that your curve $E$ has a rational
    subgroup of order 3.  Basically by inspection, you can prove that if it
    has a rational subgroup of order 5 as well, then it can't be
    semistable.  (You look at the four non-cuspidal rational points of
    $X_0(15)$.)  So you can assume that $E[5]$ is ``nice''.
    Then the idea is to
    find an $E^\prime$ with the same 5-division structure, for which 
     $E^\prime[3]$ is
    modular.  (Then $E^\prime$
     is modular, so $E^\prime[5] = E[5]$ is modular.)  You
    consider the modular curve $X$ which parameterizes elliptic curves whose
    5-division points look like $E[5]$.  This is a twist of $X(5)$.  It's
    therefore of genus 0, and it has a rational point (namely, $E$), so it's
    a projective line.  Over that you look at the irreducible covering
    which corresponds to some desired 3-division structure.  You use
    Hilbert irreducibility and the Cebotarev density theorem (in some way
    that hasn't yet sunk in) to produce a non-cuspidal rational point of $X$
    over which the covering remains irreducible.  You take $E^\prime$ to be the
    curve corresponding to this chosen rational point of $X$.
    
    
    \item {\bf From the previous version of the FAQ:}
    
    (b) conjectures arising from the study of elliptic curves and
    modular forms. -- The Taniyama-Weil-Shimura conjecture.
     
    There is a very important and well known conjecture known as the
    Taniyama-Weil-Shimura conjecture that concerns elliptic curves.
    This conjecture has been shown by the work of Frey, Serre, Ribet,
    et. al. to imply FLT uniformly, not just asymptotically as with the
    ABC conjecture.
    
    The conjecture basically states that all elliptic curves can be
    parameterized in terms of modular forms. 

    There is new work on the arithmetic of elliptic curves. Sha, the
    Tate-Shafarevich group on elliptic curves of rank 0 or 1. By the way
    an interesting aspect of this work is that there is a close 
    connection between Sha, and some of the classical work on FLT. For
    example, there is a classical proof that uses infinite descent to
    prove FLT for $n = 4$. It can be shown that there is an elliptic curve
    associated with FLT and that for $n=4$, Sha is trivial. It can also be
    shown that in the cases where Sha is non-trivial, that 
    infinite-descent arguments do not work; that in some sense ``Sha
    blocks the descent''. Somewhat more technically, Sha is an
    obstruction to the local-global principle [e.g. the Hasse-Minkowski
    theorem].

    \item {\bf From Karl Rubin:}

    \begin{teo}
 If $E$ is a semistable elliptic curve defined over $Q$,
      then $E$ is modular.
\end{teo}

    It has been known for some time, by work of Frey and Ribet, that
    Fermat follows from this.  If $u^q + v^q + w^q = 0$, then Frey had
    the idea of looking at the (semistable) elliptic curve
    $y^2 = x(x-a^q)(x+b^q)$.  If this elliptic curve comes from a modular
    form, then the work of Ribet on Serre's conjecture shows that there
    would have to exist a modular form of weight 2 on $\Gamma_0(2)$.  But
    there are no such forms.
    
    To prove the Theorem, start with an elliptic curve $E$, a prime $p$ and let
         \[\rho_p : Gal(\bar{Q}/Q) \rightarrow GL_2(Z/pZ)\]
    be the representation giving the action of Galois on the $p$-torsion
    $E[p]$.  We wish to show that a {\em certain} lift of this representation
    to $GL_2(Z_p)$ (namely, the $p$-adic representation on the Tate module
    $T_p(E)$) is attached to a modular form.  We will do this by using
    Mazur's theory of deformations, to show that {\em every} lifting which
    ``looks modular'' in a certain precise sense is attached to a modular form.
    
    Fix certain ``lifting data'', such as the allowed ramification,
    specified local behavior at $p$, etc. for the lift. This defines a
    lifting problem, and Mazur proves that there is a universal
    lift, i.e. a local ring $R$ and a representation into $GL_2(R)$ such
    that every lift of the appropriate type factors through this one.
    
    Now suppose that $\rho_p$ is modular, i.e. there is {\em some} lift
    of $\rho_p$ which is attached to a modular form.  Then there is
    also a hecke ring $T$, which is the maximal quotient of $R$ with the
    property that all {\em modular} lifts factor through $T$.  It is a
    conjecture of Mazur that $R = T$, and it would follow from this
    that {\em every} lift of $\rho_p$ which ``looks modular'' (in particular the
    one we are interested in) is attached to a modular form.
    
    Thus we need to know 2 things:
\begin{enumerate}
      \item[(a)]  $\rho_p$ is modular
      \item[(b)]  $R = T$.
\end{enumerate}
    
    It was proved by Tunnell that $\rho_3$ is modular for every elliptic
    curve.  This is because $PGL_2(Z/3Z) = S_4$.  So (a) will be satisfied
    if we take $p=3$.  This is crucial.
    
    Wiles uses (a) to prove (b) under some restrictions on $\rho_p$.  Using
    (a) and some commutative algebra (using the fact that $T$ is Gorenstein,
    basically due to Mazur)  Wiles reduces the statement $T = R$ to
    checking an inequality between the sizes of 2 groups.  One of these
    is related to the Selmer group of the symmetric square of the given
    modular lifting of $\rho_p$, and the other is related (by work of Hida)
    to an $L$-value.  The required inequality, which everyone presumes is
    an instance of the Bloch-Kato conjecture, is what Wiles needs to verify.
    
    [This is the part that turned out to be wrong in the first version]. 
    He does this using a Kolyvagin-type Euler system argument.  This is
    the most technically difficult part of the proof, and is responsible
    for most of the length of the manuscript.  He uses modular
    units to construct what he calls a {\it geometric Euler system} of
    cohomology classes.  The inspiration for his construction comes
    from work of Flach, who came up with what is essentially the
    bottom level of this Euler system.  But Wiles needed to go much
    farther than Flach did.  In the end, {\em under certain hypotheses} 
    on $\rho_p$
    he gets a workable Euler system and proves the desired inequality.
    Among other things, it is necessary that $\rho_p$ is irreducible.
    
    [The new proof replaces the argument above with one using
    commutative algebra and and some clever observations by De Shalit
    to fill in the gap.]

    Suppose now that $E$ is semistable.
    
    \noindent Case 1.  $\rho_3$ is irreducible.\\
\noindent
    Take $p=3.$  By Tunnell's theorem (a) above is true.  Under these
    hypotheses the argument above works for $\rho_3$, so we conclude
    that $E$ is modular.
    
    \noindent Case 2.  $\rho_3$ is reducible.
    Take $p=5$.  In this case $\rho_5$ must be irreducible, or else $E$
    would correspond to a rational point on $X_0(15)$.  But $X_0(15)$
    has only 4 noncuspidal rational points, and these correspond to
    non-semistable curves.  If we knew that $\rho_5$ were modular,
    then the computation above would apply and $E$ would be modular.
    
    We will find a new semistable elliptic curve $E^\prime$ such that
    $\rho_{E,5} = \rho_{E^\prime,5}$ and $\rho_{E^\prime,3}$ is 
    irreducible.  Then by Case I, $E^\prime$ is modular.
     Therefore $\rho_{E,5}= \rho_{E^\prime,5}$
    does have a modular lifting and we will be done.
    
    We need to construct such an $E^\prime$.  Let $X$ denote the modular
    curve whose points correspond to pairs $(A, C)$ where $A$ is an
    elliptic curve and $C$ is a subgroup of $A$ isomorphic to the group
    scheme $E[5]$.  (All such curves will have mod-5 representation
    equal to $\rho_E$.)  This $X$ is genus 0, and has one rational point
    corresponding to $E$, so it has infinitely many.  Now Wiles uses a
    Hilbert Irreducibility argument to show that not all rational
    points can be images of rational points on modular curves
    covering $X$, corresponding to degenerate level 3 structure
    (i.e. $im(\rho_3) \neq GL_2(Z/3)$).  In other words, an $E^\prime$ of the
    type we need exists.  (To make sure $E^\prime$ is semistable, choose
    it 5-adically close to $E$.  Then it is semistable at 5, and at
    other primes because $\rho_{E^\prime,5} = \rho_{E,5}$.)
    
\end{itemize}

\subsection{If not, then what?}

    FLT is usually broken into 2 cases. The first case assumes
    $(abc,n) = 1$. The second case is the general case.

    \subsubsection{What has been proved}
   

    First Case.

    It has been proven true up to $7.568*10^{17}$ by the work of Wagstaff \&
    Tanner, Granville \& Monagan, and Coppersmith. They all used extensions
    of the Wiefrich criteria and improved upon work performed by
    Gunderson and Shanks \& Williams.

    The first case has been proven to be true for an infinite number of
    exponents by Adelman, Frey, et. al. using a generalization of the
    Sophie Germain criterion


    \noindent Second Case:

    It has been proven true up to $n = 150,000$ by Tanner \& Wagstaff. The
    work used new techniques for computing Bernoulli numbers mod $p$ and
    improved upon work of Vandiver. The work involved computing the
    irregular primes up to 150,000. FLT is true for all regular primes
    by a theorem of Kummer. In the case of irregular primes, some
    additional computations are needed. More recently, 
    Fermat's Last Theorem has been proved true up to exponent 4,000,000
    in the general case. The method used was essentially that of Wagstaff:
    enumerating and eliminating irregular primes by Bernoulli number
    computations. The computations were performed on a set of NeXT
    computers by Richard Crandall et al.

    Since the genus of the curve $a^n + b^n = 1$, is greater than or equal
    to 2 for $n > 3$, it follows from Mordell's theorem [proved by
    Faltings], that for any given $n$, there are at most a finite number
    of solutions.


    \subsubsection{Conjectures}

    There are many open conjectures that imply FLT. These conjectures
    come from different directions, but can be basically broken into
    several classes: (and there are interrelationships between the
    classes)
\begin{enumerate}

    \item Conjectures arising from Diophantine approximation theory such
    as the ABC conjecture, the Szpiro conjecture, the Hall conjecture,
    etc.

    For an excellent survey article on these subjects see the article
    by Serge Lang in the Bulletin of the AMS, July 1990 entitled
    ``Old and new conjectured Diophantine inequalities".

    Masser and Osterle formulated the following known as the ABC
    conjecture:

    Given $\epsilon > 0$, there exists a number $C(\epsilon)$ such that for
    any set of non-zero, relatively prime integers $a,b,c$ such that
    $a+b = c$ we have
    $\max( |a|, |b|, |c|) \leq C(\epsilon) N(abc)^{1 + \epsilon}$
    where $N(x)$ is the product of the distinct primes dividing $x$.

    It is easy to see that it implies FLT asymptotically. The conjecture
    was motivated by a theorem, due to Mason that essentially says the
    ABC conjecture is true for polynomials.

    The ABC conjecture also implies Szpiro's conjecture [and vice-versa]
    and Hall's conjecture. These results are all generally believed to
    be true.

    There is a generalization of the ABC conjecture [by Vojta] which is
    too technical to discuss but involves heights of points on
    non-singular algebraic varieties . Vojta's conjecture also implies
    Mordell's theorem [already known to be true]. There are also a
   number of inter-twined conjectures involving heights on elliptic
    curves that are related to much of this stuff. For a more complete
    discussion, see Lang's article.

    \item Conjectures arising from the study of elliptic curves and
    modular forms. -- The Taniyama-Weil-Shimura conjecture.

    There is a very important and well known conjecture known as the
    Taniyama-Weil-Shimura conjecture that concerns elliptic curves.
    This conjecture has been shown by the work of Frey, Serre, Ribet,
    et. al. to imply FLT uniformly, not just asymptotically as with the
    ABC conj.

    The conjecture basically states that all elliptic curves can be
    parameterized in terms of modular forms.

    There is new work on the arithmetic of elliptic curves. Sha, the
    Tate-Shafarevich group on elliptic curves of rank 0 or 1. By the way
    an interesting aspect of this work is that there is a close
    connection between Sha, and some of the classical work on FLT. For
    example, there is a classical proof that uses infinite descent to
    prove FLT for $n = 4$. It can be shown that there is an elliptic curve
    associated with FLT and that for $n=4$, Sha is trivial. It can also be
    shown that in the cases where Sha is non-trivial, that
    infinite-descent arguments do not work; that in some sense 'Sha
    blocks the descent'. Somewhat more technically, Sha is an
    obstruction to the local-global principle [e.g. the Hasse-Minkowski
    theorem].



    \item Conjectures arising from some conjectured inequalities involving
    Chern classes and some other deep results/conjectures in arithmetic
    algebraic geometry.

    This results are quite deep. Contact
    Barry Mazur [or Serre, or Faltings, or Ribet, or ...]. Actually the
    set of people who DO understand this stuff is fairly small.

    The Diophantine and elliptic curve conjectures all involve deep
    properties of integers. Until these conjecture were tied to FLT,
    FLT had been regarded by most mathematicians as an isolated problem;
    a curiosity. Now it can be seen that it follows from some deep and
    fundamental properties of the integers. [not yet proven but
    generally believed].

\end{enumerate}
}

\subsection{Related Conjectures}

 A related conjecture from Euler

\[ x^n + y^n + z^n = c^n \mbox{ has no solution if n is} \geq 4 \]

Noam Elkies gave a counterexample, namely $2682440^4 + 15365639^4 +
18796760^4 = 20615673^4$. Subsequently, Roger Frye found the
absolutely smallest solution by (more or less) brute force: it is
$95800^4 + 217519^4 + 414560^4 = 422481^4$.  "Several years",
Math. Comp. 51 (1988) 825-835.
 

   This synopsis is quite brief. A full survey would run too many pages.

    \Ref

    \article{[1] J.P.Butler, R.E.Crandall,\& R.W.Sompolski,}
            {Irregular Primes to One Million.}
            {Math. Comp.,}
            {59 (October 1992) pp. 717-722.}

    \book{Fermat's Last Theorem, A Genetic Introduction to Algebraic Number 
          Theory.}
         {H.M. Edwards.}
         {Springer Verlag, New York, 1977.}

    \book{Thirteen Lectures on Fermat's Last Theorem.}
         {P. Ribenboim.}
         {Springer Verlag, New York, 1979.}

    \book{Number Theory Related to Fermat's Last Theorem.}
         {Neal Koblitz, editor.}
         {Birkh\"auser Boston, Inc., 1982, ISBN 3-7643-3104-6}





\subsection{Did Fermat prove this theorem?}


No he did not. Fermat claimed to have found a proof of the theorem
at an early stage in his career. Much later he spent time and effort
proving the cases $n=4$ and $n=5$. Had he had a proof to his theorem 
earlier, there would have been no need for him to study specific cases.

Fermat may have had one of the following ``proofs'' in mind when
he wrote his famous comment.

\begin{itemize}
\item Fermat discovered and applied the method of infinite descent, which,
in particular can be used to prove FLT for $n=4$.  This method can
actually be used to prove a stronger statement than FLT for $n=4$, viz,
$x^4 + y^4 = z^2$ has no non-trivial integer solutions.
  It is possible and even likely that he had an incorrect proof of FLT
using this method when he wrote the famous ``theorem''.
\item He had a wrong proof in mind. The following proof, proposed first
by Lame' was thought to be correct, until Liouville pointed out the
flaw, and by Kummer which latter became and expert in the field.
It is based on the {\em incorrect} assumption that prime decomposition is
unique in all domains.

The incorrect proof goes something like this:

We only need to consider prime exponents (this is true).
So consider $x^p + y^p = z^p$.  Let $r$ be a primitive $p$-th root of unity 
(complex number)

Then the equation is the same as:

\[(x+y)(x+ry)(x+r^2y)...(x+r^{p-1}y) = z^p\]

Now consider the ring of the form:

\[a_1 + a_2 r + a_3 r^2 + ... + a_{p-1}   r^{p-1}\] 

where each $a_i$  is an integer

Now {\bf if} this ring is a unique factorization ring (UFR), then it is true
that each of the above factors is relatively prime.

From this it can be proven that each factor is a $p$th power from
which FLT follows.  This is usually done by considering two cases, the
first where $p$ divides none of $x$, $y$, $z$; the second where $p$
divides some of $x$, $y$, $z$.  For the first case, if $x+yr=u*t^p$,
where $u$ is a unit in $Z[r]$ and $t$ is in $Z[r]$, it follows that
$x=y (mod p)$. Writing the original equation as $x^p + (-z)^p =
(-y)^p$, it follows in a similar fashion that $x = -z (mod p)$.  Thus
$2*x^p = x^p + y^p = z^p = -x^p (mod p)$ which implies $3*x^p = 0 (mod
p)$ and from there $p$ divides one of $x$ or $3|x$. But $p>3$ and $p$
does not divides $x$; contradiction.  The second case is harder.


The problem is that the above ring is {\bf not} an UFR in general.

\end{itemize}

 Another argument for the belief that Fermat had no proof ---and,
 furthermore, that he {\bf knew} that he had no proof--- is that the
only place he ever mentioned the result was in that marginal comment in
Bachet's Diophantus. If he really thought he had a proof, he would have
announced the result publicly, or challenged some English mathematician
to prove it. It is likely that he found the flaw in his own proof before
he had a chance to announce the result, and never bothered to erase the
marginal comment because it never occurred to him that anyone would see
it there.

Some other famous mathematicians have speculated on this question.
Andre Weil, writes:
\begin{quote}
Only on one ill-fated occasion did Fermat ever mention
a curve of higher genus $x^n+y^n=z^n$, and then hardly remains any
doubt that this was due to some misapprehension on his part [$\ldots$]
 for a brief moment perhaps [$\ldots$] he must have deluded himself
into thinking he had the principle of a general proof.
\end{quote}

Winfried Scharlau and Hans Opolka report:
\begin{quote}
Whether
Fermat knew a proof or not has been the subject of many speculations.  The
truth seems obvious $\ldots$ [Fermat's marginal note] was made at the time of his
first letters concerning number theory [1637]$\ldots$ as far as we know he never
repeated his general remark, but repeatedly made the statement for the
cases $n=3$ and $4$ and posed these cases as problems to his correspondents
[$\ldots$] he formulated the case $n=3$ in a letter to Carcavi in 1659
[$\ldots$]  All these
facts indicate that Fermat quickly became aware of the incompleteness of
the [general] ``proof" of 1637.  Of course, there was no reason for a public
retraction of his privately made conjecture.
\end{quote}



However it is important to keep in mind that Fermat's ``proof"
 predates the Publish or
Perish period of scientific research in which we are still living.


\Ref

  \book{From Fermat to Minkowski: lectures on the theory of numbers and its
          historical development.}{ Winfried Scharlau, Hans Opolka.}
{New York, Springer, 1985.}

\book{Basic Number Theory.}{Andre Weil.}{Berlin, Springer, 1967}










