\section{Why is there no Nobel in mathematics?}

Nobel prizes were created by the will of Alfred Nobel, a notable Swedish
chemist.

One of the most common---and unfounded---reasons as to why Nobel decided
against a Nobel prize in math is that [a woman he proposed to/his
wife/his mistress] [rejected him because of/cheated him with] a famous
mathematician. Gosta Mittag-Leffler is often claimed to be the guilty
party.

There is no historical evidence to support the story.

For one, Mr. Nobel was never married.

There are more credible reasons as to why there is no Nobel prize in
math. Chiefly among them is simply the fact he didn't care much for
mathematics, and that it was not considered a practical science from
which humanity could benefit (a chief purpose for creating the Nobel
Foundation).

Further, at the time there existed already a well known Scandinavian
prize for mathematicians. If Nobel knew about this prize he may have
felt less compelled to add a competing prize for mathematicians in his
will.

\begin{quote}
  [...] As professor ordinarius in Stockholm, Mittag-Leffler began a
  30-year career of vigorous mathematical activity. In 1882 he founded
  the Acta Mathematica, which a century later is still one of the
  world's leading mathematical journals. Through his influence in
  Stockholm he persuaded King Oscar II to endow prize competitions and
  honor various distinguished mathematicians all over Europe. Hermite,
  Bertrand, Weierstrass, and Poincare were among those honored by the
  King. [...]
\end{quote}
Source: ``The Mathematics of Sonya Kovalevskaya'' by Roger Cooke
(Springer-Verlag, New York etc., 1984, II.5.2, p. 90-91:


Here are some relevant facts:

\begin{itemize}
  \item Nobel never married, hence no ``wife''. (He did have a mistress,
  a Viennese woman named Sophie Hess.)

  \item Gosta Mittag-Leffler was an important mathematician in Sweden in
  the late 19th-early 20th century.  He was the founder of the journal
  Acta Mathematica, played an important role in helping the career of
  Sonya Kovalevskaya, and was eventually head of the Stockholm Hogskola,
  the precursor to Stockholms Universitet.  However, it seems highly
  unlikely that he would have been a leading candidate for an early
  Nobel Prize in mathematics, had there been one---there were guys like
  Poincar\'e and Hilbert around, after all.

  \item There is no evidence that Mittag-Leffler had much contact with
  Alfred Nobel (who resided in Paris during the latter part of his
  life), still less that there was animosity between them for whatever
  reason.  To the contrary, towards the end of Nobel's life
  Mittag-Leffler was engaged in ``diplomatic'' negotiations to try to
  persuade Nobel to designate a substantial part of his fortune to the
  Hogskola. It seems hardly likely that he would have undertaken this if
  there was prior bad blood between them.  Although initially Nobel
  seems to have intended to do this, eventually he came up with the
  Nobel Prize idea---much to the disappointment of the Hogskola, not to
  mention Nobel's relatives and Fraulein Hess.

  \item According to the very interesting study by Elisabeth Crawford,
  ``The Beginnings of the Nobel Institution'', Cambridge Univ. Press,
  1984, pages 52-53:
  \begin{quote}
    Although it is not known how those in responsible positions at the
    Hogskola came to believe that a large bequest was forthcoming, this
    indeed was the expectation, and the disappointment was keen when it
    was announced early in 1897 that the Hogskola had been left out of
    Nobel's final will in 1895.  Recriminations followed, with both
    Pettersson and Arrhenius [academic rivals of Mittag-Leffler in the
    administration of the Hogskola] letting it be known that Nobel's
    dislike for Mittag-Leffler had brought about what Pettersson termed
    the `Nobel Flop'.  This is only of interest because it may have
    contributed to the myth that Nobel had planned to institute a prize
    in mathematics but had refrained because of his antipathy to
    Mittag-Leffler or---in another version of the same story---because
    of their rivalry for the affections of a woman....
  \end{quote}

  However, Sister Mary Thomas a Kempis discovered a letter by R. C.
  Archibald in the archives of Brown University and discussed its
  contents in ``The Mathematics Teacher'' (1966, pp.667-668).  Archibald
  had visited Mittag-Leffler and, on his report, it would seem that M-L
  *believed* that the absence of a Nobel Prize in mathematics was due to
  an estrangement between the two men.  (This at least is the natural
  reading, but not the only possible one.)

  \item A final speculation concerning the psychological element.  Would
  Nobel, sitting down to draw up his testament, presumably in a mood of
  great benevolence to mankind, have allowed a mere personal grudge to
  distort his idealistic plans for the monument he would leave behind?

\end{itemize}

Nobel, an inventor and industrialist, did not create a prize in
mathematics simply because he was not particularly interested in
mathematics or theoretical science.  His will speaks of prizes for those
``inventions or discoveries'' of greatest practical benefit to mankind.
(Probably as a result of this language, the physics prize has been
awarded for experimental work much more often than for advances in
theory.)

However, the story of some rivalry over a woman is obviously much more
amusing, and that's why it will probably continue to be repeated.

\Ref

\article{}{}{Mathematical Intelligencer,}{vol. 7 (3), 1985, p. 74.}

\book{The Beginnings of the Nobel Institution.}{Elisabeth Crawford.}
{Cambridge Univ. Press, 1984.}
