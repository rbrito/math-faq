\section{The Axiom of Choice}


 There are several equivalent formulations:

\begin{itemize}

    \item The Cartesian product of nonempty sets is nonempty, even
    if the product is of an infinite family of sets.

    \item Given any set $S$ of mutually disjoint nonempty sets, there is a set $C$
    containing a single member from each element of $S$.  $C$ can thus be
    thought of as the result of ``choosing'' a representative from each
    set in $S$. Hence the name.
\end{itemize}

\subsection{Relevance of the Axiom of Choice}
THE AXIOM OF CHOICE

There are many equivalent statements of the Axiom of Choice.  The following
version gave rise to its name:
\begin{quote}
  For any set $X$ there is a function $f$, with domain
  $X\backslash{0}$, so that $f(x)$ is a member of $x$ for every
  nonempty $x$ in $X$.
\end{quote}
Such an $f$ is called a ``choice function'' on $X$.  [Note that $X\backslash
 {0}$ means
$X$ with the empty set removed.  Also note that in Zermelo-Fraenkel set
theory all mathematical objects are sets so each member of $X$ is itself
a set.]

The Axiom of Choice (AC) is one of the most discussed axioms of mathematics,
perhaps second only to Euclid's parallel postulate.  The axioms of set theory
provide a foundation for modern mathematics in the same way that Euclid's five
postulates provided a foundation for Euclidean geometry, and the questions
surrounding AC are the same as the questions that surrounded Euclid's Parallel
Postulate:
\begin{enumerate}
 \item  Can it be derived from the other axioms?
 \item Is it consistent with the other axioms?
 \item Should we accept it as an axiom?
\end{enumerate}
For many sets, including any finite set, the first six axioms of set theory
(abbreviated ZF) are enough to guarantee the existence of a choice function
but there do exist sets for which AC is {\em required} to show the existence of
a choice function.  The existence of such sets was proved in 1963 by Paul
Cohen.  This means that AC cannot be derived from the other six axioms;
in other words ``AC is independent of ZF.''  This answers question [1] posed
above.

The question of whether AC is consistent with the other axioms (question [2]
above) was answered by Goedel in 1938.  Goedel showed that if the other axioms
are consistent then AC is consistent with them.  This is a ``relative
consistency'' proof which is the best we can hope for because of Goedel's
Second Incompleteness Theorem.

The third question, ``Should we accept it as an axiom?'', moves us into the
realm of philosophy.  Today there are three major schools of thought
concerning the use of AC:
\begin{enumerate}
\item Accept it as an axiom and use it without hesitation.
\item Accept it as an axiom but use it only when you cannot find a proof
    without it.
\item AC is unacceptable.
\end{enumerate}
Most mathematicians today belong to school A.  Mathematicians who are in
school B are usually there because of a belief in Occam's Razor (use as few
assumptions as possible when explaining something) or an interest in
metamathematics.  There are a growing number of people moving to school C,
especially computer scientists who work on automated reasoning using
constructive type theories.

Underlying the schools of thought about the use of AC are views about truth
and the nature of mathematical objects.  Three major views are platonism,
constructivism, and formalism.

\medskip
\noindent{\bf \large Platonism}
\smallskip

A platonist believes that mathematical objects exist independent of the human
mind, and a mathematical statement, such as AC, is objectively either true or
false.  A platonist accepts AC only if it is objectively true, and probably
falls into school A or C depending on her belief.  If she isn't sure about
AC's truth then she may be in school B so that once she finds out the truth
about AC she will know which theorems are true.

\medskip
\noindent{\bf \large Constructivism}
\smallskip

A constructivist believes that the only acceptable mathematical objects are
ones that can be constructed by the human mind, and the only acceptable proofs
are constructive proofs.  Since AC gives no method for constructing a choice
set constructivists belong to school C.

\medskip
\noindent{\bf \large Formalism}
\smallskip

A formalist believes that mathematics is strictly symbol manipulation and any
consistent theory is reasonable to study.  For a formalist the notion of truth
is confined to the context of mathematical models, e.g., a formalist would say
``The parallel postulate is false in Riemannian geometry.'' but she wouldn't say
``The parallel postulate is false.''  A formalist will probably not allign
herself with any school.  She will comfortably switch between A, B, and C
depending on her current interests.


So: Should you accept the Axiom of Choice?  Here are some arguments
for and against it.

\medskip
\noindent{\bf\large Against}

\begin{itemize}
\item It's not as simple, aesthetically pleasing, and intuitive as the other
  axioms.
\item It is equivalent to many statements which are not intuitive such as ``Every
  set can be well ordered.''  How, for example, would you well order the reals?
\item With it you can derive non-intuitive results, such as the existence of a
  discontinuous additive function, the existence of a non-measurable set of
  reals, and the Banach-Tarski Paradox (see the next section of the sci.math
  FAQ).
\item It is nonconstructive---it conjures up a set without providing any sort of
  procedure for its construction.
\end{itemize}

\noindent{\bf\large For}
\smallskip

The acceptance of AC is based on the belief that our intuition about finite
sets can be extended to infinite sets.  The main argument for accepting it
is that it is useful.  Many important, intuitively plausible theorems are
equivalent to it or depend on it.  For example these statements are equivalent
to AC:
\begin{itemize}
\item Every vector space has a basis.
\item Trichotomy of Cardinals: For any cardinals $k$ and $l$, either $k<l$ or $k=l$ or $k>l$.
\item Tychonoff's Theorem: The product of compact spaces is compact in the
  product topology.
\item Zorn's Lemma: Every nonempty partially ordered set P in which each
  chain has an upper bound in P has a maximal element.
\end{itemize}
And these statements depend on AC (i.e., they cannot be proved in ZF
without AC):
\begin{itemize}
\item The union of countably many countable sets is countable.
\item Every infinite set has a denumerable subset.
\item The Loewenheim-Skolem Theorem: Any first-order theory which has a model
  has a denumerable model.
\item The Baire Category Theorem: The reals are not the union of countably
  many nowhere dense sets (i.e., the reals are not meager).
\item The Ultrafilter Theorem: Every Boolean algebra has an ultrafilter on it.
\end{itemize}

\noindent{\bf\large Alternatives to AC}

\begin{itemize}
\item Accept only a weak form of AC such as the Denumerable Axiom of Choice
  (every denumerable set has a choice function) or the Axiom of Dependent
  Choice.
\item Accept an axiom that implies AC such as the Axiom of Constructibility
  ($V=L$) or the Generalized Continuum Hypothesis (GCH).
\item Adopt AC as a logical axiom (Hilbert suggested this with his epsilon
  axiom).  If set theory is done in such a logical formal system the
  Axiom of Choice will be a theorem.
\item Accept a contradictory axiom such as the Axiom of Determinacy.
\item Use a completely different framework for mathematics such as
  Category Theory.  Note that within the framework of Category Theory
  Tychonoff's Theorem can be proved without AC (Johnstone, 1981).
\end{itemize}

\noindent{\bf\large Test Yourself: When is AC necessary?}
\smallskip

If you are working in Zermelo-Fraenkel set theory without the Axiom of
Choice, can you choose an element from...
\begin{enumerate}
\item  a finite set?
 \item an infinite set?
 \item each member of an infinite set of singletons (i.e., one-element sets)?
 \item each member of an infinite set of pairs of shoes?
 \item each member of inifinite set of pairs of socks?
 \item each member of a finite set of sets if each of the members is infinite?
 \item each member of an infinite set of sets if each of the members is infinite?
 \item each member of a denumerable set of sets if each of the members is infinite?
 \item each member of an infinite set of sets of rationals?
\item each member of a denumerable set of sets if each of the members is
    denumberable?
\item each member of an infinite set of sets if each of the members is finite?
\item each member of an infinite set of finite sets of reals?
\item each member of an infinite set of sets of reals?
\item each member of an infinite set of two-element sets whose members are
    sets of reals?
\end{enumerate}
The answers to these questions with explanations are accessible through
\url{http://www.jazzie.com/ii/math/index.html}.


\Ref

Benacerraf, Paul and Putnam, Hilary.  ``Philosophy of Mathematics: Selected
Readings'', 2nd edition. Cambridge University Press, 1983.

Dauben, Joseph Warren.  ``Georg Cantor: His Mathematics and Philosophy of the
Infinite.''  Princeton University Press, 1979.

A. Fraenkel, Y.  Bar-Hillel, and A. Levy with van Dalen, Dirk. ``Foundations
of Set Theory,'' Second Revised Edition. North-Holland, 1973.

Johnstone, Peter T.  ``Tychonoff's Theorem without the Axiom of Choice.''
Fundamenta Mathematica 113: 21-35, 1981.

Leisenring, Albert C.  ``Mathematical Logic and Hilbert's Epsilon-Symbol.''
Gordon and Breach, 1969.

Maddy, ``Believing the Axioms, I'', J. Symb. Logic, v. 53, no. 2, June 1988,
pp. 490-500, and ``Believing the Axioms II'' in v.53, no. 3.

Moore, Gregory H.  ``Zermelo's Axiom of Choice: Its Origins, Development, and
Influence.''  Springer-Verlag, 1982.

Rubin, Herman and Rubin, Jean E.  ``Equivalents of the Axiom of Choice II.''
North-Holland, 1985.

This section of the FAQ is Copyright (c) 1994 Nancy McGough.  Send comments
and or corrections relating to this part to \url{nancym@ii.com}.  The most up to
date version of this section of the sci.math FAQ is accesible through
\url{http://www.jazzie.com/ii/math/index.html}.
