\subsection{Why is $0.9999\ldots = 1$?}

In modern mathematics, the string of symbols $0.9999\ldots $ is
understood to be a shorthand for ``the infinite sum $9/10 + 9/100 +
9/1000 + \ldots$''.  This in turn is shorthand for ``the limit of the
sequence of real numbers $9/10$, $9/10 + 9/100$, $9/10 + 9/100 + 9/1000,
\ldots$''. Using the well-known epsilon-delta definition of the limit
(you can find it in any of the given references on analysis), one can
easily show that this limit is $1$.  The statement that $0.9999\ldots =
1$ is simply an abbreviation of this fact.

\[
0.9999\ldots = \sum_{n=1}^{\infty}\frac{9}{10^n} = \lim_{m\rightarrow
  \infty}\sum_{n=1}^m\frac{9}{10^n}
\]

Choose $\varepsilon > 0$. Suppose $\delta = 1/-\log_{10} \varepsilon$,
thus $\varepsilon = 10^{-1/\delta}$. For every $m>1/\delta$ we have that

\[
\left| \sum_{n=1}^m \frac{9}{10^n} - 1 \right| = \frac{1}{10^m} <
\frac{1}{10^{1/\delta}}=\varepsilon
\]

So by the $\varepsilon-\delta$ definition of the limit we have

\[
\lim_{m\rightarrow \infty}\sum_{n=1}^m\frac{9}{10^n}=1
\]

Not formal enough? In that case you need to go back to the construction
of the number system. After you have constructed the reals (Cauchy
sequences are well suited for this case, see [Shapiro75]), you can
indeed verify that the preceding proof correctly shows $0.9999\ldots =
1$.

An informal argument could be given by noticing that the following
sequence of ``natural'' operations has as a consequence $0.9999\ldots =
1$. Therefore it's ``natural'' to assume $0.9999\ldots = 1$.

\begin{eqnarray*}
      x &=& 0.9999\ldots \\
    10x &=& 10 \cdot 0.9999\ldots \\
    10x &=& 9.9999\ldots \\
10x - x &=& 9.9999\ldots - 0.9999\ldots \\
     9x &=& 9 \\
      x &=& 1 \\
\end{eqnarray*}

Thus $0.9999\ldots = 1$.

% Credit: kdq@professor.jpl.nasa.gov (Kevin D. Quitt) 941107
%         He gave the following argument.
An even easier argument multiplies both sides of $0.3333\ldots = 1/3$ by
$3$.  The result is $0.9999\ldots = 3/3 = 1$.

Another informal argument is to notice that all periodic numbers such as
$0.46464646\ldots$ are equal to the period divided over the same number
of $9$s. Thus $0.46464646\ldots=46/99$. Applying the same argument to
$0.9999\ldots=9/9=1$.

Although the three informal arguments might convince you that
$0.9999\ldots = 1$, they are not complete proofs. Basically, you need to
prove that each step on the way is allowed and is correct. They are also
``clumsy'' ways to prove the equality since they go around the bush:
proving $0.9999\ldots = 1$ directly is much easier.

You can even have that while you are proving it the ``clumsy'' way, you
get proof of the result in another way. For instance, in the first
argument the first step is showing that $0.9999\ldots$ is real
indeed. You can do this by giving the formal proof stated in the
beginning of this FAQ question. But then you have $0.9999\ldots = 1$ as
corollary. So the rest of the argument is irrelevant: you already proved
what you wanted to prove.

\Ref

\book{R.V. Churchill and J.W. Brown.} {Complex Variables and
  Applications.} {$5^{th}$ ed., McGraw-Hill, 1990.}

\book{E. Hewitt and K. Stromberg.} {Real and Abstract Analysis.}
{Springer-Verlag, Berlin, 1965.}

\book{W. Rudin.} {Principles of Mathematical Analysis.} {McGraw-Hill,
  1976.}

\book{L. Shapiro.} {Introduction to Abstract Algebra.} {McGraw-Hill,
  1975.}

\article{F. Richman.} {Is 0.999 = 1?,} {Mathematics Magazine,} {72
  (1999), 404--408.}
%%% Local Variables:
%%% mode: latex
%%% TeX-master: "math-faq"
%%% End:
