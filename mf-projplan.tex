%Brendan McKay. Not true any more.

\section{Projective plane of order 10}

More precisely:

Is it possible to define 111 sets (lines) of 11 points each such that:

For any pair of points there is precisely one line containing them both
and for any pair of lines there is only one point common to them both?


Analogous questions with $n^2 + n + 1$ and $n + 1$ instead of 111 and 11
have been positively answered only in case n is a prime power.  For
$n=6$ it is not possible, more generally if $n$ is congruent to 1 or 2
mod 4 and can not be written as a sum of two squares, then an FPP of
order $n$ does not exist.  The $n=10$ case has been settled as not
possible either by Clement Lam. As the ``proof'' took several years of
computer search (the equivalent of 2000 hours on a Cray-1) it can be
called the most time-intensive computer assisted single proof. The final
steps were ready in January 1989.

\Ref

\article{R. H. Bruck and H. J. Ryser.}  {The nonexistence of certain
  finite projective planes.}  {Canadian Journal of Mathematics,} {vol. 1
  (1949), pp 88-93.}

\article{C. Lam.}{}{American Mathematical Monthly,}{98 (1991), 305-318.}
%%% Local Variables:
%%% mode: latex
%%% TeX-master: "math-faq"
%%% End:
