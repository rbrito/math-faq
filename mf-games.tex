\section{The Monty Hall problem}

This problem has rapidly become part of the mathematical folklore.

The American Mathematical Monthly, in its issue of January 1992,
explains this problem carefully. The following are excerpted from that
article.

Problem:

A TV host shows you three numbered doors (all three equally likely), one
hiding a car and the other two hiding goats. You get to pick a door,
winning whatever is behind it. Regardless of the door you choose, the
host, who knows where the car is, then opens one of the other two doors
to reveal a goat, and invites you to switch your choice if you so
wish. Does switching increases your chances of winning the car?

If the host always opens one of the two other doors, you should switch.
Notice that $1/3$ of the time you choose the right door (i.e. the one
with the car) and switching is wrong, while $2/3$ of the time you choose
the wrong door and switching gets you the car.

Thus the expected return of switching is $2/3$ which improves over your
original expected gain of $1/3$.

Even if the hosts offers you to switch only part of the time, it pays to
switch.  Only in the case where we assume a malicious host (i.e. a host
who entices you to switch based in the knowledge that you have the right
door) would it pay not to switch.

There are several ways to convince yourself about why it pays to switch.
Here's one. You select a door. At this time assume the host asks you if
you want to switch {\bf before} he opens any doors. Even though the odds
that the door you selected is empty are high ($2/3$), there is no
advantage on switching as there are two doors, and you don't know thich
one to switch to. This means the $2/3$ are evenly distributed, which as
good as you are doing already.  However, once Monty opens one of the two
doors you selected, the chances that you selected the right door are
still $1/3$ and now you only have one door to choose from if you
switch. So it pays to switch.


\Ref

\article{L. Gillman}{The Car and the Goats} {American Mathematical
  Monthly,}{January 1992, pp. 3-7.}


\section{Master Mind}

For the game of Master Mind it has been proven that no more than five
moves are required in the worst case.

One such algorithm was published in the Journal of Recreational
Mathematics; in '70 or '71 (I think), which always solved the $4$ peg
problem in $5$ moves. Knuth later published an algorithm which solves
the problem in a shorter number of moves---on average---but can take six
guesses on certain combinations.

In 1994, Kenji Koyama and Tony W. Lai found, by exhaustive search that
$5625/1296~=4.340$ is the optimal strategy in the expected case. This
strategy may take six guesses in the worst case.  A strategy that uses
at most five guesses in the worst case is also shown. This strategy
requires $5626/1296~=4.341$ guesses.


\Ref

\article{Donald E. Knuth.}  {The Computer as Master Mind.}
{J. Recreational Mathematics,} { 9 (1976-77), 1-6.}

\article{Kenji Koyama, Tony W. Lai.}  {An optimal Mastermind Strategy.}
{J. Recreational Mathematics,} {Vol. 25, Number 3, pp.251-256.}
%%% Local Variables: 
%%% mode: latex
%%% TeX-master: "math-faq"
%%% End: 
