\section{Unsolved Problems}

\subsection{Does there exist a number that is perfect and odd?}

    A given number is perfect if it is equal to the sum of all its proper
    divisors. This question was first posed by Euclid in ancient Greece.
    This question is still open.  Euler proved that if  $N$  is an odd
    perfect number, then in the prime power decomposition of $N$, exactly
    one exponent is congruent to 1 mod 4 and all the other exponents are
    even. Furthermore, the prime occurring to an odd power must itself be
    congruent to 1 mod 4.  A sketch of the proof appears in Exercise 87,
    page 203 of Underwood Dudley's Elementary Number Theory.
    It has been shown that there are no odd perfect numbers $< 10^{300}$.


\subsection{Collatz Problem}


       Take any natural number $m > 0$.\\
       $n:=m;$\\
       repeat\\
             \hspace{1cm}  if ($n$ is odd) then $n:=3*n+1$; else $n:=n/2$;\\
       until ($n==1$)\\

\begin{conj}  For all positive integers $m$, the program above terminates.
\end{conj}

The conjecture has been verified for all numbers up to $5.6 * 10^{13}$.



\Ref


\book{Unsolved Problems in Number Theory.}
     {Richard K Guy.}
     {Springer, Problem E16.}
\book{Elementary Number Theory.}
     {Underwood Dudley.}
     {2nd ed.}
\article{G.T. Leavens and M. Vermeulen}
 {3x+1 search programs}]
 {Comput. Math. Appl.}
{vol. 24 n. 11 (1992), 79-99.}


%From: Saouter Yannick <Yannick.Saouter@irisa.fr>
%Date: Wed, 04 Dec 1996 17:18:45 +0100
%
%@Article{Leavens,
  %author =      {G.T. Leavens and M. Vermeulen},
  %title =       {3x+1 search programs},
  %journal =     {Comput. Math. Appl.},
  %year =        {1992},
  %volume =      {24},
  %number =      {11},
  %pages =       {79-99}
%}
 

\subsection{Goldbach's conjecture}

This conjecture claims that every even integer bigger equal to 4 is
expressible as the sum of two prime numbers.
It has been tested for all values up to $4.10^{10}$
by Sinisalo.

%From: Saouter Yannick <Yannick.Saouter@irisa.fr>
%Date: Thu, 28 Nov 1996 09:09:45 +0100
% 
% 
%You precise that Goldbach's conjecture has been verified up to
%2*10^10. It is a work of te Riele and al. and there is a more
%recent work of Sinisalo:
% 
%@article{Sinisalo,
%author= {{Sinisalo}, M.K.},
%title= {Checking the {G}oldbach Conjecture up to $4.10^{11}$},
%journal= {Math. Comp.},
%number={204},
%volume={61},
%pages={931-934},
%year= {1993}
%}

%From: David G Radcliffe <radcliff@csd.uwm.edu>
%Date: Mon, 27 Nov 1995 12:31:55 -0600 (CST)
%Conversely, it is known that Goldbach conjecture is true for all
%numbers larger than $e^{e^{16038}}$

\subsection{Twin primes conjecture}

There exist an infinite number of positive integers $p$ with
        $p$ and $p+2$ both prime. See the largest known twin prime
section. There are some results on the estimated density of twin
primes.

