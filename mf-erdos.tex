\section{Erdos Number}

    Form an undirected graph where the vertices are academics, and an
    edge connects academic $X$ to academic $Y$ if $X$ has written a paper
    with $Y$. The Erdos number of $X$ is the length of the shortest path
    in this graph connecting $X$ with Erdos.

    Erdos has Erdos number 0.  Co-authors of Erdos have Erdos number 1.
    Einstein has Erdos number 2, since he wrote a paper with Ernst Straus,
    and Straus wrote many papers with Erdos.


    The Extended Erdos Number applies to co-authors of Erdos.
    For People who have authored more than one paper with Erdos,
    their Erdos number is defined to be $1/\#$ papers-co-authored.
  %  Ron Graham has the smallest, non-zero, Erdos number.

    Why people care about it?

     Nobody seems to have a reasonable answer...

    Who is Paul Erdos?

    Paul Erdos was an Hungarian mathematician. He obtained his PhD
    from the University of Manchester and spent most of his
    efforts tackling "small" problems and conjectures related to
    graph theory, combinatorics, geometry and number theory.

    He was one of the most prolific publishers of papers; and was
    also and indefatigable traveller.

    Paul Erd\"os died on September 20, 1996.

    At this time the number of people with Erdos number 2 or less
    is estimated to be over 4750, according to Professor Jerrold
    W. Grossman archives. These archives can be consulted via
    anonymous ftp at vela.acs.oakland.edu under the directory
    pub/math/erdos or on the Web at http://www.acs.oakland.edu/~grossman/erdoshp.html.
    At this time it contains a list of all co-authors
    of Erdos and their co-authors.

    On this topic, he writes

\begin{quote}
Let $E_1$ be the subgraph of the collaboration graph induced by people with
Erd\H os number~1. We found that $E_1$ has 451 vertices and 1145 edges.
Furthermore, these collaborators tended to collaborate a lot,
especially among themselves. They have an average of 19 other collaborators
(standard deviation~21), and only seven of them collaborated with no one
except Erd\H os. Four of them have over 100 co-authors. If we restrict our
attention just to~$E_1$, we still find a lot of joint work. Only 41 of
these 451 people have collaborated with no other persons with Erd\H os
number~1 (i.e., there are 41 isolated vertices in~$E_1$), and $E_1$ has
four components with two vertices each. The remaining 402 vertices in $E_1$
induce a connected subgraph. The average vertex degree in $E_1$ is~5, with
a standard deviation of~6; and there are four vertices with degrees of 30
or higher. The largest clique in $E_1$ has seven vertices, but it should be
noted that six of these people and Erd\H os have a joint seven-author
paper. In addition, there are seven maximal 6-cliques, and 61 maximal
5-cliques. In all, 29 vertices in $E_1$ are involved in cliques of order 5
or larger. Finally, we computed that the diameter of $E_1$
is 11 and its radius is~6.

Three quarters of the people with Erd\H os number~2 have only one co-author
with Erd\H os number~1 (i.e., each such person has a unique path of length 2
to~$p$). However, their mean number of Erd\H os number~1 co-authors is
1.5, with a standard deviation of~1.1, and the count ranges as high as~13.

Folklore has it that most active researchers have a finite, and fairly
small, Erd\H os number. For supporting evidence, we verified that all the
Fields and Nevanlinna prize winners during the past three cycles
(1986--1994) are indeed in the Erd\H os component, with Erd\H os number at
most~9. Since this group includes people working in theoretical physics,
one can conjecture that most physicists are also in the Erd\H os component,
as are, therefore, most scientists in general. The large number
of applications of graph theory to the social sciences might also lead one
to suspect that many researchers in other academic areas are included as
well. We close with two open questions about~$C$, restricted to
mathematicians, that such musings suggest, with no hope that either will ever
be answered satisfactorily: What is the diameter of the Erd\H os
component, and what is the order of the second largest component?
\end{quote}

   \Ref

    \article{Caspar Goffman.}{And what is your Erdos number?}
    {American Mathematical Monthly,}{v. 76 (1969), p. 791.}

   \article{Tom Odda (alias for Ronald Graham)} {On Properties of a
   Well- Known Graph, or, What is Your Ramsey Number?}
   { Topics in Graph Theory}{ (New York, 1977), pp.  [166-172].}


