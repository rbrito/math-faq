\section{Prime Numbers}

\subsection{Largest known Mersenne prime}

Mersenne primes are primes of the form $2^p-1$. For $2^p-1$ to be prime
we must have that $p$ is prime.

$2^{6972593}-1$ is prime. It was discovered in 1999.

\subsection{Largest known prime}

The largest known prime is the Mersenne prime described above.  The
largest known non-Mersenne prime, is $391581*2^{216193}-1$, discovered
by Brown, Noll, Parady, Smith, Smith, and Zarantonello.

Throughout history, the largest known prime has almost always been a
Mersenne prime; the period between Brown et al's discovery in August
1989 and Slowinski \&\ Gage's in March 1992 is one of the few
exceptions.

%From: Bill Stewart <billstewart@mail.att.net>
%Date: Mon, 06 Jan 1997 00:15:11 -0800
You can help find more primes. For more information see: The Great
Internet Mersenne Prime Search home page on
\url{http://www.mersenne.org}

\Ref

\article{Brown, Noll, Parady, Smith, Smith, and Zarantonello.}{Letter to
  the editor.}{American Mathematical Monthly,}{vol. 97, 1990, p. 214.}

\subsection{Largest known twin primes}

% From: Karl-Heinz Indlekofer <k-heinz@mathematik.uni-paderborn.de>
% From: "Chris K. Caldwell" <caldwell@UTM.Edu>
% Date: Wed, 8 Nov 1995 10:32:59 +0100

The two largest known twin primes are $361700055 * 2^39020 \pm 1$.  with
11755 digits, found by Lifchitz in 1999.
%From Warut Roonguthai <kamala@chulkn.car.chula.ac.th>
%Date: Sat, 30 Dec 1995 23:07:07 +0700 (BKK)
%They are also the first known gigantic twin primes (primes with at
%least 10,000 digits).

%Recent record holders are:
%
%\begin{itemize}
%From: Harvey Dubner <70372.1170@compuserve.com> via
%		 Warut Roonguthai <kamala@chulkn.car.chula.ac.th>
%Date: Thu, 7 Dec 1995 15:42:55 +0700 (BKK)
%\item $190116*3003*10^{5120} \pm 1$, with 5129 digits, by Harvey Dubner.
%\item $697053813 * 2^{16352} \pm 1$, with 4932 digits, found by Indlekofer
%and Ja'rai in 1994.
%\item $1691232 * 1001 * 10^{4020} \pm 1$  with 4030 digits, found by H. Dubner.
%\item $4650828 * 1001 * 10^{3429}\pm 1$.  Found by H. Dubner as well.
%\end{itemize}

Same as Harvey Dubner above.  The two largest Sophie Germain primes
(i.e. $p$ and $2p+1$ are both primes) are $p=2687145 * 3003 * 10^{5072}
- 1$ and $q=2p + 1$, found by Harvey Dubner, in October 3, 1995.

%   $  p = 157324389 * 2^{16352} - 1 $ and $p = 470943129 * 2^{16352} - 1$,
%with 4932 digits each, also found by Indlekofer and Jarai, in 1994-1995.

\Ref

\article{B. K. Parady and J. F. Smith and S. E. Zarantonello, Smith,
  Noll and Brown.}{ Largest known twin primes.}  { Mathematics of
  Computation,}{ vol.55, 1990, pp. 381-382. }


\subsection{Largest Fermat number with known factorization}

$F_{11} = (2^{2^{11}}) + 1$ which was factored by Brent \& Morain in
1988. $F_9 = (2^{2^9}) + 1 = 2^{512} + 1$ was factored by A.K. Lenstra,
H.W. Lenstra Jr., M.S. Manasse \& J.M. Pollard in
1990. %The factorization for $F_{10}$ is not known.
$F_{10}$ was factored by Richard Brent who found a 40-digit factor of
$2^{1024} + 1$ on October 20, 1995. The cofactor is a 252 digit number,
which is not so easy to factor. Luckily, this number was also prime.

\subsection{Algorithms to factor integer numbers}

There are several known algorithms that have subexponential estimated
running time, to mention just a few:

\begin{itemize}
  \item Continued fraction algorithm.
  \item Quadratic sieve algorithm.
  \item Class Group method.
  \item Elliptic curve algorithm.
  \item Number field sieve.
  \item Dixon's random squares algorithm.
  \item Valle's two-thirds algorithm.
  \item Seysen's class group algorithm.
\end{itemize}


\Ref

\article{A.K. Lenstra, H.W. Lenstra Jr.}{Algorithms in Number Theory.}
{J. van Leeuwen (ed.), Handbook of Theoretical Computer Science, Volume
  A: Algorithms and Complexity}{ Elsevier, pp.  673-715, 1990.}

\subsection{Primality Testing}

The problem of primality testing and factorization are two distinct
problems. If we concentrate on primality testing, we never need to know
the actual factors. The only question to be answered is ``is the number
in question prime or composite.''

Wilson's Theorem: The integer $p$ is prime if and only if $(p-1)!$ is
congruent to $-1 \pmod p$

Since there is no known method for rapidly computing $(N-1)! \pmod N$
in, say, $\log N$ steps, so Wilson's characterization of primes is of no
practical value to the testing of the primality of $N$.

There are many different primality tests and they can be classified in
at least three different ways:

\begin{enumerate}
  \item Tests for numbers of special forms \\
  versus \\
  Tests for generic numbers
  \item Tests with full justification \\
  versus\\
  Tests with justification based on conjectures
  \item Deterministic tests\\
  versus \\
  Probabilistic or Monte Carlo tests
\end{enumerate}

{\bf Miller's Test}

In 1976, G. L. Miller proposed a primality test, which was justified
using a generalized form of Riemann's hypothesis.

{\bf The APR Test}

The primality test devised by L. M. Adleman, C. Pomerance and R. S.
Rumely (1983), also known as the APR test, represents a breakthrough
because:

\begin{enumerate}
  \item It is applicable to arbitrary natural numbers $N$, without
  requiring the knowledge of factors of $N - 1$ or $N + 1$.
  \item The running time $t(N)$ is almost polynomial.
  \item The test is justified rigorously, and for the first time ever in
  this domain, it is necessary to appeal to deep results in the theory
  of algebraic numbers; it involves calculations with roots of unity and
  the general reciprocity law for the power residue symbol.
\end{enumerate}

The running time of the APR is at the present the world record for a
deterministic primality test.

Soon afterwards, H. Cohen \& A. K. Lenstra (1984) modified the APR test,
making it more flexible, using Gauss sums in the proof (instead of the
reciprocity law), and having the new test programmed for practical
applications. It was the first primality test in existence that can
routinely handle numbers of up 100 decimal digits, and it does so in
about 45 seconds.

{\bf Monte Carlo methods}

Ribenboim mentions three Monte Carlo tests, due to R. Baillie \&
Wagstaff, Jr. (1980), R. Solovay \& V. Strassen (1977), and M. O. Rabin
(1976, 1980).

{\bf Elliptic Curves Primality Proving, ECPP}

ECPP stands for ``Elliptic Curves and Primality Proving''. The algorithm
is described in:

\begin{verbatim}
    A. O. L. Atkin and F. Morain
    "Elliptic curves and primality proving"
    To appear.
\end{verbatim}

It is a deterministic algorithm that gives a certificate of primality
for numbers that can be as large as 10**1000 (one thousand).

References

\begin{verbatim}
[1] A. O. L. Atkin and F. Morain
    "Elliptic curves and primality proving"
    To appear in Math. Comp.

\end{verbatim}
% Lieven Marchand <mal@bewoner.dma.be>

\begin{verbatim}
[2] F. Morain
    "Courbes elliptiques et tests de primalite'"
    The`se, Universite' de Lyon I, 1990.
    Available at:
http://lix.polytechnique.fr/~morain/english-index.html
\end{verbatim}

This subsection is copyright (C) 1995. Harry J. Smith,
\url{HJSmith@ix.netcom.com}.

\subsection{List of record numbers}

Chris Caldwell (\url{caldwell@utm.edu}) maintains a list called ``The
Largest Known Primes.''  Some of the ways to get this list are:

\begin{itemize}
  \item web: \url{http://www.utm.edu/research/primes/largest.html}
  \item ftp: \url{ftp://math.utm.edu/pub/math/primes}
\end{itemize}

Finger \url{primes@math.utm.edu} for a few record primes and the current
ways to get the lists.  He would like to know of any new titanic primes
(over 1000 digits) so that he can add them to his list.


\subsection{What is the current status on Mersenne primes?}

The following Mersenne primes are known.

\bigskip
\bigskip
\hspace{2cm}
\begin{tabular}{|r|r|r|l|} \hline

{Number} & {$p$} & {Year}& {Discoverer} \\ \hline

     1-4 &   2,3,5,7      &             pre-1500 &\\
     5   &      13        &                1461 & Anonymous \\
     6-7 &      17,19        &             1588 & Cataldi\\
     8   &       31          &             1750 & Euler\\
     9   &       61          &             1883 & I.M. Pervushin\\
    10   &       89          &             1911 & Powers\\
    11   &       107         &             1914 & Powers\\
    12   &       127         &             1876 & Lucas\\
    13-14&       521,607     &             1952 & Robinson\\
    15-17&       1279,2203,2281  &         1952 & R. M. Robinson\\
    18   &       3217            &         1957 & Riesel\\
    19-20&       4253,4423       &         1961 & Hurwitz \& Selfridge \\
    21-23&       9689,9941,11213 &         1963 & Gillies\\
    24   &       19937           &         1971 & Tuckerman\\
    25   &       21701           &         1978 & Noll \& Nickel\\
    26   &       23209           &         1979 & Noll\\
    27   &       44497           &         1979 & Slowinski \& Nelson\\
    28   &       86243           &         1982 & Slowinski\\
    29   &       110503          &         1988 & Colquitt \& Welsh\\
    30   &       132049          &         1983 & Slowinski\\
    31   &       216091          &         1985 & Slowinski\\
    32   &       756839          &         1992 & Slowinski \& Gage\\
    33  &       859433          &         1994 & Slowinski \& Gage\\
    34  &      1257787          &         1996 & Slowinski \& Gage \\
    35  &      1398269          &         1996 & Armengaud, Woltman, et. al.\\
    36  &      2976221          &         1997 & Spence, Woltman, et. al.\\
    37 & 	3021377		&		1998 & Clarkson, Woltman, Kurowski, GIMPS \\
    38? & 	6972593		&		1999 & Nayan Hajratwala, GIMPS \\
\hline
\end{tabular}


\bigskip
\bigskip

The way to determine if $2^p-1$ is prime is to use the Lucas-Lehmer
test, described in Algorithm~\ref{alg:lucaslehmer}.

\newcommand{\lucaslehmer}{\ensuremath{\mbox{\sc Lucas-Lehmer}}}
\begin{algorithm}[H]
  \caption{$\lucaslehmer(n)$}\label{alg:lucaslehmer}
  \begin{algorithmic}[1]

    \REQUIRE A prime integer $p$.
    \ENSURE Prints if $2^p-1$ is prime or not.

    \smallskip

    \STATE $u \leftarrow 4$
    \FOR{$i = 3, \ldots, p$}
        \STATE $u \leftarrow u^2-2 \bmod 2^p-1$
    \ENDFOR
    \IF {$u = 0$}
       \PRINT $2^p-1$ is prime
    \ELSE
       \PRINT $2^p-1$ is composite
    \ENDIF
  \end{algorithmic}
\end{algorithm}

All exponents less than 1,481,800 have now been tested at least once.

\Ref

\book{An introduction to the theory of numbers.}{G.H. Hardy,
  E.M. Wright.}  {Fifth edition, 1979, Oxford.}

\subsection{Formulae to compute prime numbers}


There is no polynomial which gives all the prime numbers. This is a
simple exercise to prove.

There is no non-constant polynomial that only takes on prime values.
The proof is simple enough that an high school student could probably
discover it.  See, for example, Ribenboim's book {\it The Book of Prime
  Number Records.}

Note, however, by the work of Jones, Sato, Wada, and Wiens, there {\it
  is} a polynomial in 26 variables such that the set of primes coincides
with the set of {\it positive} values taken by this polynomial.  See
Ribenboim, pp. 147--150.

But most people would object to the term ``formula'' restricted to mean
polynomial.  Can we not use summation signs, factorial, and the floor
function in our ``formula''?  If so, then indeed, there {\it are}
formulas for the prime numbers.  Some of them are listed below.

A reasonable interpretation of the word ``formula'' is simply ``Turing
machine that halts on all inputs''.  Under this interpretation, there
certainly are halting Turing machines which compute the $n$-th prime
number.  However, nobody knows how to compute the $n$-th prime in time
polynomial in $\log n$.  That's still an open question.

Herb Wilf has addressed the question, ``What is a formula?'' in his
article, ``What is an answer?'' which appeared in the American
Mathematical Monthly, 89 (1982), 289-292.  He draws a distinction
between ``formula'' and ``good formula''.  Anyone who claims ``there is
no formula for the prime numbers'' should read this article.

Here are just a few articles that discuss ``formulas'' for primes.
Almost all of these do {\bf not} require computation of the primes ahead
of time.  Most of them rely on standard mathematical functions such as
summation, factorial, greatest integer function, etc.


\Ref

\article{C. Isenkrahe.}{}{Math. Annalen}{53 (1900), 42-44.}

\article{W. H. Mills.}{}{Bulletin of the American Mathematical Society}{
  53 (1947), 604.}

\article{L. Moser.}{}{Mathematics Magazine}{23 (1950), 163-164.}

\article{E. M. Wright.}{}{American Mathematical Monthly}{58 (1951),
  616-618.  (Correction, 59 (1952), 99.)}

\article{E. M. Wright.}{}{Journal of the London Mathematical Society}{29
  (1954), 63-71.}

\article{B. R. Srinivasan.}{}{Journal of the Indian Mathematical
  Society}{25 (1961), 33-39.}

\article{C. P. Willans.}{}{Mathematics Gazette}{48 (1964), 413-415.}

\article{V. C. Harris.}{}{Nordisk Mat. Tidskr.}{17 (1969), 82.}

\article{U. Dudley.}{}{American Mathematical Monthly}{76 (1969), 23-28.}

\article{C. Vanden Eynden.}{}{American Mathematical Monthly} {79 (1972),
  625.}

\article{S. W. Golomb.}{}{American Mathematical Monthly} {81 (1974),
  752-754.}

\medskip

\book{Algorithmic Number Theory.}{J.O. Shallit, E. Bach.}  {(to be
  published, MIT Press).}  \book{A Course in Computational Algebraic
  Number Theory.}{Henri Cohen.}{ Springer-Verlag, Graduate Texts in
  Math, 1993.}
%%% Local Variables:
%%% mode: latex
%%% TeX-master: "math-faq"
%%% End:
