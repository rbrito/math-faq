 First a brief explanation: In the Gregorian Calendar, over a period
    of four hundred years, there are 97 leap years and 303 normal years.
    Each normal year, the day of January 1 advances by one; for each leap
    year it advances by two.

     \[   303 + 97 + 97 = 497 = 7 * 71 \]

    As a result, January 1 year $N$ occurs on the same day of the week as
    January 1 year $N + 400$.  Because the leap year pattern also recurs
    with a four hundred year cycle, a simple table of four hundred
    elements, and single modulus, suffices to determine the day of the
    week (in the Gregorian Calendar), and does it much faster than all the
    other algorithms proposed.  Also, each element takes (in principle)
    only three bits; the entire table thus takes only 1200 bits; 
    %Date: Mon, 20 Nov 1995 08:51:23 +0100
    %From: Juergen Weinelt <rzuw039@rz.uni-wuerzburg.de>
    %1200/8=150
    %
    %or 300 bytes; 
    on many computers this will be less than the instructions to do
    all the complicated calculations proposed for the other algorithms.

    Incidental note: Because 7 does not divide 400, January 1 occurs more
    frequently on some days than others!  Trick your friends!  In a cycle
    of 400 years, January 1 and March 1 occur on the following days with
    the following frequencies:

\begin{verbatim}
           Sun      Mon     Tue     Wed     Thu     Fri     Sat
    Jan 1: 58       56      58      57      57      58      56
    Mar 1: 58       56      58      56      58      57      57
\end{verbatim}

    Of interest is that (contrary to most initial guesses) the occurrence
    is not maximally flat.

    % I saw the article in the Gazette. I remember having seen this 
    % result before. AL-O
    % Also ray johnstone <ray@iinet.com.au> pointed it out on Tue, 11 Apr 1995

    In the Mathematical Gazette, vol. 53,, pp.127-129, it is shown that 
    the 13th of the month is more likely to be a Friday than any other
    day.The author is a 13 year old S.R.Baxter.


    The Gregorian calendar was introduced in 1582 in parts of Europe; it was
    adopted in 1752 in Great Britain and its colonies, and on various dates
    in other countries.  It replaced the Julian Calendar which has a four-year
    cycle of leap years; after four years January 1 has advanced by five days.
    Since 5 is relatively prime to 7, a table of $4 * 7 = 28$ elements is
    necessary for the Julian Calendar.


    There is still a 3 day over 10,000 years error which the Gregorian calendar
    does not take into account.  At some time such a correction will have
    to be done but your software will probably not last that long!

    Here is a standard method suitable for mental computation:


\begin{enumerate}
        \item Take the last two digits of the year.
        \item Divide by 4, discarding any fraction.
        \item Add the day of the month.
        \item Add the month's key value: JFM AMJ JAS OND
                                         144 025 036 146
        \item Subtract 1 for January or February of a leap year.
        \item For a Gregorian date, add 0 for 1900's, 6 for 2000's,
             4 for 1700's,
           2 for 1800's; for other years, add or subtract multiples of 400.
        \item For a Julian date, add 1 for 1700's, and 1 for every additional
           century you go back.
        \item Add the last two digits of the year.
        \item Divide by 7 and take the remainder.
\end{enumerate}

        Now 1 is Sunday, the first day of the week, 2 is Monday, and so on.

    The following formula, which is for the Gregorian calendar only, may be
    more convenient for computer programming.  Note that in some programming
    languages the remainder operation can yield a negative result if given
    a negative operand, so mod 7 may not translate to a simple remainder.
       \[ W = (k + \lfloor 2.6m - 0.2\rfloor - 2C + Y + \lfloor Y/4
\rfloor + \lfloor C/4\rfloor) mod 7 \]
           where $\lfloor\ \rfloor$  denotes the integer floor function,\\
           $k$ is day (1 to 31)\\
           $m$ is month (1 = March, ..., 10 = December, 11 = Jan, 12 = Feb)
                         Treat Jan \& Feb as months of the preceding year\\
           $C$ is century (1987 has $C$ = 19)\\
           $Y$ is year    (1987 has $Y$ = 87 except $Y$ = 86 for Jan \& Feb)\\
           $W$ is week day (0 = Sunday, ..., 6 = Saturday)\\

    Here the century and 400 year corrections are built into the formula.
    The $\lfloor 2.6m-0.2\rfloor$ term relates to the repetitive pattern that the 30-day
    months show when March is taken as the first month.

The following short C program works for a restricted range, it returns 0 for
Monday, 1 for Tuesday, etc.

\begin{verbatim}
dow(m,d,y){y-=m<3;return(y+y/4-y/100+y/400+"-bed=pen+mad."[m]+d)%7;}
\end{verbatim}

The program appeared  was posted by
\url{sakamoto@sm.sony.co.jp} (Tomohiko Sakamoto) on comp.lang.c on March
10th, 1993.

A good mnemonic rule to help on the computation of the day of the week is
as follows. In any given year the following days come on
the same day of the week:

\begin{verbatim}
4/4
6/6
8/8
10/10
12/12
\end{verbatim}

\noindent to remember the next four, remember that I work from 9--5 at a 7--11 so

\begin{verbatim}
9/5
5/9
7/11
11/7
\end{verbatim}

\noindent and the last day of Feb.

``In 1995 they come on Tuesday. Every year this advances one other than
leap-years which advance 2. Therefore for 1996 the day will be
Thursday, and for 1997 it will be Friday. Therefore ordinarily every 4
years it advances 5 days. There is a minor correction for the century
since the century is a leap year iff the century is divisible by 4.
Therefore 2000 is a leap year, but 1900, 1800, and 1700 were not.''

Even ignoring the pattern over for a period of years this is still
useful since you can generally figure out what day of the week a given
date is on faster than someone else can look it up with a calender if
the calender is not right there. (A useful skill that.)


    \Ref

    \book{Winning Ways for your mathematical plays.}{Elwyn
R. Berlekamp, John H. Conway, and Richard K. Guy}
    {London ; Toronto : Academic Press, 1982.}

    \book{Mathematical Carnival.}{Martin Gardner.}{New York : Knopf, c1975.}

    \book{Elementary Number Theory and its applications.}{Kenneth Rosen.}
    {Reading, Mass. ; Don Mills, Ont. : Addison-Wesley Pub. Co., c1993. p. 156.}


    \article{Michael Keith and Tom Craver.}{The Ultimate Perpetual Calendar?}
    {Journal of Recreational Mathematics,}{22:4, pp. 280-282, 19}
