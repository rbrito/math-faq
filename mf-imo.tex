\section{International Mathematics Olympiad and Other Competitions}

From the IMO home page:

The International Mathematics Olympiad (IMO) is an annual mathematics 
competition for highschool students. It is one of the International 
Science Olympiads. The first IMO was held in Romania in 1959. The
usual size of an official delegation to an IMO is (a maximum of) 
six student competitors and (a maximum of) two leaders. There is no 
official ``team''. The student competitors write two papers, on 
consecutive days, each paper consisting of three questions. Each question 
is worth seven marks. 

You can check results and other info at
\url{http://www.win.tue.nl/win/ioi/imo/}.
