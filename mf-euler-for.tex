\section{Euler's formula: $e^{i \pi} = - 1 $}

The definition and domain of exponentiation has been changed several
times.  The original operation $x^y$ was only defined when $y$ was a
positive integer.  The domain of the operation of exponentation has been
extended, not so much because the original definition made sense in the
extended domain, but because there were (almost) unique ways to extend
exponentation which preserved many of what seemed to be the
``important'' properties of the original operation.  So in part, these
definitions are only convention, motivated by reasons of aesthetics and
utility.

The original definition of exponentiation is, of course, that $x^y = 1
\times x \times x \times \cdots \times x,$ where $1$ is multiplied by
$x$, $y$ times.  This is only a reasonable definition for $y=1, 2, 3,
\dots$ (It could be argued that it is reasonable when $y=0$, but that
issue is taken up in a different part of the FAQ).  This operation has a
number of properties, including
\begin{enumerate}
  \item $x^1 = x$
  \item For any $x$, $n$, $m$, $x^n x^m = x^{n+m}$.
  \item If $x$ is positive, then $x^n$ is positive.

  Now, we can try to see how far we can extend the domain of
  exponentiation so that the above properties (and others) still hold.
  This naturally leads to defining the operation $x^y$ on the domain
  {$x$ positive real; $y$ rational}, by setting $x^{p/q} =$ the $q^{th}$
  root of $x^p$.  This operation agrees with the original definition of
  exponentiation on their common domain, and also satisfies (1), (2) and
  (3).  In fact, it is the unique operation on this domain that does so.
  This operation also has some other properties:

  \item If $x>1$, then $x^y$ is an increasing function of $y$.
  \item If $0<x<1$, then $x^y$ is a decreasing function of $y$.

  Again, we can again see how far we can extend the domain of
  exponentiation while still preserving properties (1)-(5).  This leads
  naturally to the following definition of $x^y$ on the domain {$x$
    positive real; $y$ real}:

  If $x > 1$, $x^y$ is defined to be $\sup_q\{x^q\}$, where $q$ runs
  over all rationals less than or equal to $y$.

  If $x < 1$, $x^y$ is defined to be $\inf_q\{x^q\}$, where $q$ runs
  over all rationals bigger than or equal to $y$.

  If $x = 1$, $x^y$ is defined to be $1$.

  Again, this operation satisfies (1)--(5), and is in fact the only
  operation on this domain to do so.

  The next extension is somewhat more complicated.  As can be proved
  using the methods of calculus or combinatorics, if we define $e$ to be
  the number
  \[
  e = 1 + 1/1! + 1/2! + 1/3! + \cdots = 2.71828...
  \]
  it turns out that for every real number $x$,

  \item $ e^x = 1 + x/1! + x^2/2! + x^3/3! + \cdots$

  $e^x$ is also denoted $\exp(x)$.  (This series always converges
  regardless of the value of $x$).

  One can also define an operation $\ln x$ on the positive reals, which
  is the inverse of the operation of exponentiation by $e$.  In other
  words, $\exp(\ln x) = x$ for all positive $x$.  Moreover,

  \item If $x$ is positive, then $x^y$ = $\exp(y \ln x)$.  Because of
  this, the natural extension of exponentiation to complex exponents,
  seems to be to define
  \[
  \exp(z) = 1 + z/1! + z^2/2! + z^3/3! + \cdots
  \]

  for all complex $z$ (not just the reals, as before), and to define

  $x^z = \exp(z \ln x)$

  when $x$ is a positive real and $z$ is complex.

  This is the only operation $x^y$ on the domain {$x$ positive real, $y$
    complex} which satisfies all of (1)--(7).  Because of this and other
  reasons, it is accepted as the modern definition of exponentiation.

  From the identities
  \begin{align}
    \sin x &= x - x^3/3! + x^5/5! - x^7/7! + \cdots\\
    \cos x &= 1 - x^2/2! + x^4/4! - x^6/6! + \cdots
  \end{align}
  which are the Taylor series expansion of the trigonometric sine and
  cosine functions respectively.  From this, one sees that, for any real
  $x$,

  \item $\exp(i x) = \cos x + i \sin x.$

  Thus, we get Euler's famous formula
  \[
  e^{\pi i} = -1
  \]
  and
  $e^{2\pi i} = e^0 = 1.$

  One can also obtain the classical addition formulae for sine and
  cosine from (8) and (1).
\end{enumerate}

All of the above extensions have been restricted to a positive real for
the base.  When the base $x$ is not a positive real, it is not as
clear-cut how to extend the definition of exponentiation.  For example,
$(-1)^{1/2}$ could well be $i$ or $-i$, $(-1)^{1/3}$ could be $-1$, $1/2
+ \sqrt 3 i/2$, or $1/2 - \sqrt 3 i/2$, and so on.  Some values of $x$
and $y$ give infinitely many candidates for $x^y$, all equally
plausible.  And of course $x=0$ has its own special problems.  These
problems can all be traced to the fact that the exp function is not
injective on the complex plane, so that ln is not well defined outside
the real line.  There are ways around these difficulties (defining
branches of the logarithm, for example), but we shall not go into this
here.

The operation of exponentiation has also been extended to other systems
like matrices and operators.  The key is to define an exponential
function by (6) and work from there.  [Some reference on operator
calculus and/or advanced linear algebra?]

\Ref

\book{Complex Analysis.}  {Ahlfors, Lars V.}  {McGraw-Hill, 1953.}
%%% Local Variables:
%%% mode: latex
%%% TeX-master: "math-faq"
%%% End:
