\section{How to compute digits of $\pi$?}

Symbolic Computation software such as {\it Maple} or {\it Mathematica}
can compute $10,000$ digits of $\pi$ in a blink, and another
$20,000--1,000,000$ digits overnight (range depends on hardware
platform).

It is possible to retrieve $1.25+$ million digits of $\pi$ via anonymous
ftp from the addresses
\url{ftp://wuarchive.wustl.edu/doc/misc/pi/pi.doc.Z},
\url{ftp://wuarchive.wustl.edu/doc/misc/pi/pi.dat.Z}.  New York's
Chudnovsky brothers have computed $2$ billion digits of $\pi$ on a
homebrew computer.

The current record is held by Yasumasa Kanada and Daisuke Takahashi from
the University of Tokyo with $51$ billion digits of $\pi$
($51,539,600,000$ decimal digits to be precise).

Nick Johnson-Hill has an interesting page of $\pi$ trivia at:
\url{http://www.users.globalnet.co.uk/~nickjh/Pi.htm}.

%Modification date: Sept 4, 1995.
%
%From: "Antreas P. Hatzipolakis" <xpolakis@athena.compulink.gr>
%Date: Sun, 3 Sep 1995 15:50:24 -0400
%
%The new (June 95) world record for decimals of pi is: 3,221,220,000 digits.
%Please visit the site: ftp://www.cc.u-tokyo.ac.jp/
%
%
%    The new record for the number of digits of $\pi$ is 3,221,220,000 digits,
%which are available at
%\begin{verbatim}
% ftp://www.cc.u-tokyo.ac.jp/
%end{verbatim}


%Modification date: Sept 6, 1995
%
%From: kanada@pi.cc.u-tokyo.ac.jp (Yasumasa KANADA)
%Newsgroups: sci.math,sci.math.symbolic
%Subject: New world record of pi
%Date: 6 Sep 1995 05:28:27 GMT
The new record for the number of digits of $\pi$ is 4.29496 billion
decimal digits of pi were calculated and verified by 28th August '95.

Related documents are available from \url{ftp://www.cc.u-tokyo.ac.jp/}.

This computations were made by Yasumasa Kanada, at the University of
Tokyo.

%Date: Fri, 24 Sep 1993 13:21:44 -0400
%From: cohen@ceremab.u-bordeaux.fr (Henri Cohen)
% The following text is from Henri


There are essentially 3 different methods to calculate $\pi$ to many
decimals.

\begin{enumerate}

  \item One of the oldest is to use the power series expansion of
  $\arctan(x)=x-x^3/3+x^5/5-\ldots$ together with formulas like
  $\pi=16*\arctan(1/5)-4*\arctan(1/239)$. This gives about $1.4$ decimals per
  term.

  \item A second is to use formulas coming from Arithmetic-Geometric
  mean computations. A beautiful compendium of such formulas is given in
  the book $\pi$ and the AGM, (see references).  They have the advantage
  of converging quadratically, i.e. you double the number of decimals
  per iteration.  For instance, to obtain $1,000,000$ decimals, around
  $20$ iterations are sufficient. The disadvantage is that you need FFT
  type multiplication to get a reasonable speed, and this is not so easy
  to program.

  \item A third one comes from the theory of complex multiplication of
  elliptic curves, and was discovered by S. Ramanujan. This gives a
  number of beautiful formulas, but the most useful was missed by
  Ramanujan and discovered by the Chudnovsky's. It is the following
  (slightly modified for ease of programming):

  Set $k_1=545140134;$ $k_2=13591409;$ $k_3=640320;$ $k_4=100100025;$
  $k_5=327843840;$ $k_6=53360;$

  Then $\pi=\frac{k_6\sqrt{k_3}}{S}$, where

  \[S=\sum_{n=0}^\infty
  (-1)^n\frac{(6n)!(k_2+nk_1)}{n!^3(3n)!(8k_4k_5)^n}\]

  The great advantages of this formula are that

  1) It converges linearly, but very fast (more than $14$ decimal digits
  per term).

  2) The way it is written, all operations to compute S can be
  programmed very simply. This is why the constant $8k_4k_5$ appearing
  in the denominator has been written this way instead of
  $262537412640768000$.  This is how the Chudnovsky's have computed
  several billion decimals.
\end{enumerate}

%From: jrv@mbunix.mitre.org (Jim Van Zandt)
%Date: 7 Dec 1995 17:41:42 GMT
%Newsgroups: sci.math

%From: Robert Israel <israel@math.ubc.ca>
%Date: Sun, 19 Nov 95 00:44:22 -0800
%
An interesting new method was recently proposed by David Bailey, Peter
Borwein and Simon Plouffe. It can compute the $N$th {\bf hexadecimal}
digit of Pi efficiently without the previous $N-1$ digits. The method is
based on the formula:

\[
\pi = \sum_{i=0}^\infty {1 \over 16^i} \left(
  {4 \over 8i+1} - {2 \over 8i+4} - {1 \over 8i+5} - {1 \over 8i+6}
\right)
\]

in $O(N)$ time and $O(log N)$ space. (See references.)

The following $160$ character C program, written by Dik T. Winter at
CWI, computes $\pi$ to 800 decimal digits.

\begin{verbatim}
     int a=10000,b,c=2800,d,e,f[2801],g;main(){for(;b-c;)f[b++]=a/5;
     for(;d=0,g=c*2;c-=14,printf("%.4d",e+d/a),e=d%a)for(b=c;d+=f[b]*a,
     f[b]=d%--g,d/=g--,--b;d*=b);}
\end{verbatim}


\Ref

\article{P. B. Borwein, and D. H. Bailey.}{Ramanujan, Modular Equations,
  and Approximations to $\pi$} {American Mathematical Monthly,}{vol. 96,
  no. 3 (March 1989), p. 201-220.}

\article{D. H. Bailey, P. B. Borwein, and S. Plouffe.}{A New Formula for
  Picking off Pieces of Pi,}{Science News,} {v 148, p 279 (Oct 28,
  1995).  also at http://www.cecm.sfu.ca/\~pborwein }

\article{D. Bailey, P. Borwein and S. Plouffe.}{On the rapid computation
  of various polylogarithmic constants,}{ Math. Comp.}  { 66 (1997)
  903-913; MR 98d:11165.}

\article{J.M. Borwein and P.B. Borwein.}  {The arithmetic-geometric mean
  and fast computation of elementary functions.}  {SIAM
  Review,}{Vol. 26, 1984, pp. 351-366.}

\article{J.M. Borwein and P.B. Borwein.}  {More quadratically converging
  algorithms for $\pi$.}  {Mathematics of Computation,}{Vol. 46, 1986,
  pp. 247-253.}

\article {Shlomo Breuer and Gideon Zwas} {Mathematical-educational
  aspects of the computation of $\pi$}
{Int. J. Math. Educ. Sci. Technol.,} {Vol. 15, No. 2, 1984,
  pp. 231-244.}

\article{David Chudnovsky and Gregory Chudnovsky.}  {The computation of
  classical constants.}{Columbia University,
  Proc. Natl. Acad. Sci. USA,}{Vol. 86, 1989.}

    %reference
    % From: robert@cs.caltech.edu (Robert J. Harley)
    % Date: Thu, 30 Mar 1995 17:36:32 -0500
    % 
\book {Classical Constants and Functions: Computations and Continued
  Fraction Expansions} {D.V.Chudnovsky, G.V.Chudnovsky, H.Cohn,
  M.B.Nathanson, eds.}  {Number Theory, New York Seminar 1989-1990.}

\article{Y. Kanada and Y. Tamura.}  {Calculation of $\pi$ to 10,013,395
  decimal places based on the Gauss-Legendre algorithm and Gauss
  arctangent relation.}  {Computer Centre, University of Tokyo,}{1983.}

\article{Morris Newman and Daniel Shanks.}  {On a sequence arising in
  series for $\pi$.}  {Mathematics of computation,}{Vol. 42, No. 165,
  Jan 1984, pp. 199-217.}

\article{E. Salamin.}  {Computation of $\pi$ using arithmetic-geometric
  mean.}  {Mathematics of Computation,}{Vol. 30, 1976, pp. 565-570}

\article{David Singmaster.}  {The legal values of $\pi$.}  {The
  Mathematical Intelligencer,}{Vol. 7, No. 2, 1985.}

\article{Stan Wagon.}  {Is $\pi$ normal?}  {The Mathematical
  Intelligencer,}{Vol. 7, No. 3, 1985.}

\medskip

\book {A history of $\pi$.}  {P. Beckman.}  {Golem Press, CO, 1971
  (fourth edition 1977)}

\book {$\pi$ and the AGM - a study in analytic number theory and
  computational complexity.}  {J.M. Borwein and P.B. Borwein.}  {Wiley,
  New York, 1987.}

%%% Local Variables: 
%%% mode: latex
%%% TeX-master: "math-faq"
%%% End: 
%%% Local Variables: 
%%% mode: latex
%%% TeX-master: "math-faq"
%%% End: 
