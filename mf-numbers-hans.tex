\section{What are numbers?}

\subsection{Introduction}

Informally:
\begin{itemize}
  \item $\N = \{0,1,\ldots\}$ or $\N = \{1,2,\ldots\}$ \\
  Wether $0$ is in $\N$ depends on where you live and what is your field
  of interest. At the informal level it is a religious topic.
  \item $\Z = \{\ldots,-1,0,1,\ldots\}$
  \item $\Q = \{p/q | p, q \in \Z \mbox{ and } q \neq 0\}$
  \item $\R = \{d_0.d_1d_2\ldots | d_0 \in \Z \mbox{ and } 0 \leq d_i
  \leq 9 \mbox{ for } i>0\}$
  \item $\C = \{a+b \cdot i | a, b \in \R \mbox{ and } i^2 = -1\}$
\end{itemize}

\subsection{Construction of the Number System}

% Credit: mmcadams@cco.caltech.edu (Matt McAdams) 951209
%         main stream --> mainstream

Formally (following the mainstream in math) the numbers are constructed
from scratch out of the axioms of Zermelo Fraenkel set theory (a.k.a.\
ZF set theory) [Enderton77, Henle86, Hrbacek84]. The only things that
can be derived from the axioms are sets with the empty set at the bottom
of the hierarchy.  This will mean that any number is a set (it is the
only thing you can derive from the axioms). It doesn't mean that you
always have to use set notation when you use numbers: just introduce the
numerals as an abbreviation of the formal counterparts.

% Credit: aaron@mmml.demon.co.uk (Aaron Turner) 950330
%         He suggested numerals as an alternative for ``informal'' numbers.

% Credit: mcknighl@ix.netcom.com (Lawrence McKnight) 941104
%         He asked to add rationales to the constructions.

The construction starts with $\N$ and algebraically speaking, $\N$ with
its operations and order is quite a weak structure. In the following
constructions the structures will be strengthen one step at the time:
$\Z$ will be an integral domain, $\Q$ will be a field, for the field
$\R$ the order will be made complete, and field $\C$ will be made
algebraically complete.
% Credit: aaron@mmml.demon.co.uk (Aaron Turner) 950330
%         He asked for a brief explanation. This is as for as I'll go:
%
% [Is there a volunteer in the crowd who likes to write a section that
% explains these structures and concepts?]
%
% Yep, there is: andrea.depaoli@mail.esrin.esa.it (Andy de Paoli) 951102
% So these lines are removed and Andy's part is added by Alex to the FAQ.

Before we start, first some notational stuff:
\begin{itemize}
  \item a pair $(a,b) = \{\{a\},\{a,b\}\}$,
  \item an equivalence class $[a] = \{b | a \equiv b\}$,
  \item the successor of $a$ is $s(a) = a \cup \{a\}$.
\end{itemize}
% Credit: aaron@mmml.demon.co.uk (Aaron Turner) 950330
%         He liked to put a little more stress on the fact that other
%         definitions are possible (although this is the de facto one.

Although the previous notations and the constructions that follow are
the de facto standard ones, there are different definitions possible.

\subsection{Construction of \sl $N$}

\begin{itemize}
  \item $\{\} \in \N$
  \item if $a \in \N$ then $s(a) \in \N$
  \item $\N$ is the smallest possible set such that the preceding rules
  hold.
\end{itemize}
Informally $n=\{0,\ldots,n-1\}$ (thus $0=\{\}$, $1=\{0\}$, $2=\{0,1\}$,
$3=\{0,1,2\}$). We will refer to the elements of $\N$ by giving them a
subscript $_n$. The relation $<_n$ on $\N$ is defined as: $a_n <_n b_n$
iff $a_n \in b_n$. We can define $+_n$ as follows:
\begin{itemize}
  \item $a_n +_n 0_n = a_n$
  \item $a_n +_n s(b_n) = s(a_n +_n b_n)$
\end{itemize}
Define $*_n$ as:
\begin{itemize}
  \item $a_n *_n 0_n = 0_n$
  \item $a_n *_n s(b_n) = (a_n *_n b_n) +_n a_n$
\end{itemize}

\subsection{Construction of \sl Z}

We define an equivalence relation on $\N \times \N$: $(a_n,b_n) \equiv_z
(c_n,d_n)$ iff $a_n +_n d_n = c_n +_n b_n$. Note that $\equiv_z$
``simulates'' a subtraction in $\N$. $\Z = \{[(a_n,b_n)]_z | a_n, b_n
\in \N\}$. We will refer to the elements of $\Z$ by giving them a
subscript $_z$.  The elements of $\N$ can be embedded as follows:
$embed_n: \N \rightarrow \Z$ such that $embed_n(a_n) =
[(a_n,0_n)]_z$. Furthermore we can define:
\begin{itemize}
  \item $[(a_n,b_n)]_z <_z [(c_n,d_n)]_z$ iff $a_n +_n d_n <_n c_n +_n b_n$
  \item $[(a_n,b_n)]_z +_z [(c_n,d_n)]_z = [(a_n +_n c_n, b_n +_n d_n)]_z$
  \item $[(a_n,b_n)]_z *_z [(c_n,d_n)]_z =$ \\
        $[\left((a_n *_n c_n) +_n (b_n *_n d_n), (a_n *_n d_n) +_n (c_n *_n
          b_n)\right)]_z$
\end{itemize}

\subsection{Construction of \sl Q}

We define an equivalence relation on $\Z \times (\Z \backslash
\{0_z\})$: $(a_z,b_z) \equiv_q (c_z,d_z)$ iff $a_z *_z d_z = c_z *_z
b_z$. Note that $\equiv_q$ ``simulates'' a division in $\Z$. $\Q =
\{[(a_z,b_z)]_q | a_z \in \Z \mbox{ and } b_z \in \Z \backslash
\{0_z\}\}$. We will refer to the elements of $\Q$ by giving them a
subscript $_q$. The elements of $\Z$ can be embedded as follows:
$embed_z: \Z \rightarrow \Q$ such that $embed_z(a_z) =
[(a_z,1_z)]_q$. Furthermore we can define:
\begin{itemize}
  \item $[(a_z,b_z)]_q <_q [(c_z,d_z)]_q$ iff $a_z *_z d_z <_z c_z *_z b_z$
  when $0_z <_z b_z \mbox{ and } 0_z <_z d_z$
  \item $[(a_z,b_z)]_q +_q [(c_z,d_z)]_q = [\left((a_z *_z d_z) +_z (c_z *_z
    b_z), b_z *_z d_z\right)]_q$
  \item $[(a_z,b_z)]_q *_q [(c_z,d_z)]_q = [(a_z *_z c_z, b_z *_z d_z)]_q$
\end{itemize}

\subsection{Construction of \sl R}

The construction of $\R$ is different (and more awkward to understand)
because we must ensure that the cardinality of $\R$ is greater than that
of $\Q$.

Set $c$ is a {\em Dedekind cut} iff
\begin{itemize}
  \item $\{\} \subset c \subset \Q$ (strict inclusions!)
  \item $c$ is {\em closed downward}: \\
  if $a_q \in c$ \mbox{ and } $b_q <_q a_q$ then $b_q \in c$
  \item $c$ has no {\em largest element}: \\
      there isn't an element $a_q \in c$ such that $b_q <_q a_q$ for all $b_q
      \neq a_q \in c$
\end{itemize}

You can think of a cut as taking a pair of scissors and cutting $\Q$ in
two parts such that one part contains all the small numbers and the
other part contains all large numbers. If the part with the small
numbers was cut in such a way that it doesn't have a largest element, it
is called a Dedekind cut.  $\R = \{c | c \mbox{ is a Dedekind
  cut}\}$. We will refer to the elements of $\R$ by giving them a
subscript $_r$. The elements of $\Q$ can be embedded as follows:
$embed_q: \Q \rightarrow \R$ such that $embed_q(a_q) = \{b_q | b_q <_q
a_q\}$. Furthermore we can define:

\begin{itemize}
  \item $a_r <_r b_r$ iff $a_r \subset b_r$ (strict inclusion!)
  \item $a_r +_r b_r = \{c_q +_q d_q | c_q \in a_r \mbox{ and } d_q \in b_r\}$
  \item $-_r a_r = \\
  \{b_q |$ there exists an $c_q \in \Q$ such that $b_q <_q c_q \mbox{ and
  } (-1)_q *_q c_q \not\in a_r\}$
  \item $|a_r|_r = a_r \cup -_r a_r$
  \item $*_r$ is defined as:
  \begin{itemize}
    \item if not $a_r <_r 0_r$ and not $b_r <_r 0_r$ \\ then $a_r *_r b_r =
            0_r \cup \{c_q *_q d_q | c_q \in a_r \mbox{ and } d_q \in b_r\}$
    \item if $a_r <_r 0_r \mbox{ and } b_r <_r 0_r$ then $a_r *_r b_r =
            |a_r|_r *_r |b_r|_r$
    \item otherwise $a_r *_r b_r = -_r (|a_r|_r *_r |b_r|_r)$
  \end{itemize}
\end{itemize}

There exists an alternative definition of $\R$ using Cauchy sequences: a
Cauchy sequence is a $s: \N \rightarrow \Q$ such that $s(i_n) +_q
\left((-1)_q *_q s(j_n)\right)$ can be made arbitrary near to $0_q$ for
all sufficiently large $i_n$ and $j_n$. We will define an equivalence
relation $\equiv_r$ on the set of Cauchy sequences as: $r \equiv_r s$
iff $r(m_n) +_q \left((-1)_q *_q s(m_n)\right)$ can be made arbitrary
close to $0_q$ for all sufficiently large $m_n$. $\R = \{ [s]_r | s
\mbox{ is a Cauchy sequence}\}$.  Note that this definition is close to
``decimal'' expansions.

\subsection{Construction of \sl C}

$\C = \R \times \R$. We will refer to the elements of $\C$ by giving
them a subscript $_c$. The elements of $\R$ can be embedded as follows:
$embed_r: \R \rightarrow \C$ such that $embed_r(a_r) = (a_r,0_r)$.
Furthermore we can define:
\begin{itemize}
\item $(a_r,b_r) +_c (c_r,d_r) = (a_r +_r c_r, b_r +_r d_r$)
\item $(a_r,b_r) *_c (c_r,d_r) = \left((a_r *_r c_r) +_r -_r (b_r * d_r),
      (a_r *_r d_r) +_r (b_r *_r c_r)\right)$
\end{itemize}

There exists an elegant alternative definition using ideals. To be a bit
sloppy: $\C = \R[x] / <(x *_r x) +_r 1_r>$, i.e.\ $\C$ is the resulting
quotient ring of factoring ideal $<(x *_r x) +_r 1_r>$ out of the ring
$\R[x]$ of polynomials over $\R$. The sloppy part is that we need to
define concepts like quotient ring, ideal, and ring of polynomials. Note
that this definition is close to working with $i^2=-1$: $(x *_r x) +_r
1_r = 0_r$ can be rewritten as $(x *_r x) = (-1)_r$.

\subsection{Rounding things up}

At this moment we don't have that $\N$ is a subset of $\Z$, $\Z$ of
$\Q$, etc. But we can get the inclusions if we look at the embedded
copies of $\N$, $\Z$, etc. Let
\begin{itemize}
  \item $\N' = \mbox{ ran } embed_r \circ embed_q \circ embed_z \circ embed_n$
  \item $\Z' = \mbox{ ran } embed_r \circ embed_q \circ embed_z$
  \item $\Q' = \mbox{ ran } embed_r \circ embed_q$
  \item $\R' = \mbox{ ran } embed_r$
\end{itemize}
For these sets we have $\N' \subseteq \Z' \subseteq \Q' \subseteq \R'
\subseteq \C$. Furthermore these sets have all the properties that the
``informal'' numbers have.

\subsection{What's next?}

Well, for some of the more alien parts of math we can extend this
standard number system with some exotic types of numbers. To name a few:
\begin{itemize}
  \item Cardinals and ordinals \\
      % credit: edgar@math.ohio-state.edu (Gerald Edgar) 941104
      %         He asked to add cardinals and ordinals.
      Both are numbers in ZF set theory [Enderton77, Henle86, Hrbacek84] and
      so they are sets as well. Cardinals are numbers that represent the
      sizes of sets, and ordinals are numbers that represent well ordered
      sets. Finite cardinals and ordinals are the same as the natural
      numbers. Cardinals, ordinals, and their arithmetic get interesting and
      ``tricky'' in the case of infinite sets.
  \item Hyperreals \\
      These numbers are constructed by means of ultrafilters [Henle86] and
      they are used in non-standard analysis. With hyperreals you can treat
      numbers like Leibnitz and Newton did by using infinitesimals.
  \item Quaternions and octonions \\
      Normally these are constructed by algebraic means (like the alternative
      $\C$ definition that uses ideals) [Shapiro75, Dixon94]. Quaternions are
      used to model rotations in 3 dimensions. Octonions, a.k.a.\ Cayley
      numbers, are just esoteric artifacts :-). Well, if you know where they
      are used for, feel free to contribute to the FAQ.
      % Credit: jpb@iris85.biosym.com (Jan Bielawski) 950501
      %         Added the aka for octonions.
  \item Miscellaneous \\
      Just to name some others: algebraic numbers [Shapiro75], $p$-adic
      numbers [Shapiro75], and surreal numbers (a.k.a.\ Conway
      numbers) [Conway76].
      % Credit: aaron@mmml.demon.co.uk (Aaron Turner) 950330
      %         He asked to include Conway numbers. I've added the first two.
      % Credit: jpb@iris85.biosym.com (Jan Bielawski) 950501
      %         Added the "reverse" aka for Conway numbers.
      % Credit: jpb@iris85.biosym.com (Jan Bielawski) 950502
      %         Added reference to Conway. I gave the other ones.
\end{itemize}
Cardinals and ordinals are commonly used in math. Most mortals won't
encounter (let alone use) hyperreals, quaternions, and octonions.

\Ref

\book{J.H. Conway.}
     {On Numbers and Games, L.M.S. Monographs, vol. 6.}
     {Academic Press, 1976.}
     % Credit: jpb@iris85.biosym.com (Jan Bielawski) 950502
     %         He supplied this reference.

\book{H.B. Enderton.}
     {Elements of Set Theory.}
     {Academic Press, 1977.}

\book{G.M. Dixon.}
     {Division Algebras; Octonions, Quaternions, Complex Numbers and the
     Algebraic Design of Physics.}
     {Kluwer Academic, 1994.}

\book{J.M. Henle.}
     {An Outline of Set Theory.}
     {Springer Verlag, 1986.}

\book{K. Hrbacek and T. Jech.}
     {Introduction to Set Theory.}
     {M. Dekker Inc., 1984.}

\book{L. Shapiro.}
     {Introduction to Abstract Algebra.}
     {McGraw-Hill, 1975.}
